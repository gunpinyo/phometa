\documentclass[master.tex]{subfiles}
\begin{document}

\thispagestyle{empty}

\begin{abstract}
  Manually drawing a derivation tree usually takes many iterations to be
  completed due to its layout (its width grows exponentially to its height) and
  variables being rewritten (by unification when derivation rule is applied).
  Even when the tree is completed, there is nothing to guarantee that the tree
  is error free.

  This thesis describes \emph{Phometa} which is a proof assistant that allows a
  user to create a formal system and prove its theorems using derivation trees.
  Fundamentally, Phometa consists of three kinds of node which are
  \emph{Grammar} (Backus-Naur Form), \emph{Rule} (derivation rule), and
  \emph{Theorem} (derivation tree).

  It can be used as an educational platform for students to learn certain formal
  systems provided in a standard library. Alternatively, it also can be used as
  an experimental sandbox where a user implements their own formal system and
  tries to reason about it.

  Phometa is a web application so components such as terms and derivation trees
  can be rendered nicely in a web browser and users can interact with these
  components directly by clicking a button or pressing keyboard shortcuts.
  Visualisation also allows Phometa to have certain features that text-based
  proof assistants couldn't have, for example, nested underlines can be used to
  group terms instead of brackets, input method of terms can be controlled in
  such a way that ill from terms couldn't be created, and so on.

  Phometa has been designed and been implemented in such a way that it is
  powerful enough to completely replace a derivation-trees's manual-drawing and
  easy enough to be used by anyone. Its standard library also includes famous
  formal systems such as \emph{Simple Arithmetic}, \emph{Propositional Logic},
  and \emph{Typed Lambda Calculus}. This shows that Phometa is generic enough to
  handle most of formal systems out there.

\end{abstract}

\renewcommand{\abstractname}{Acknowledgements}
\begin{abstract}

  First and most importantly, I would like to thank my supervisor, Dr Krysia
  Broda, for her constant dedication to supporting me on both project skill and
  mental advice.

I also would like to thank my second marker, Prof. Alessio R. Lomuscio, for his
invaluable advice on the interim report.

I also would like to thank many of Imperial's computing students, for their
feedback on my project.

I also would like to thank Evan Czaplicki and the rest of Elm community, for
their ambition and effort to develop Elm --- a wonderful front-end functional
reactive language that Phometa is built on top of.

And finally, I also would like to thank my parents for their unconditional love
and continuous support on everything since I was born.

\end{abstract}

\end{document}