\documentclass[master.tex]{subfiles}
\begin{document}

\chapter{Further Examples of Formal Systems}
\label{chap:examples_formal_systems}

The last chapter successfully encoded a formal system named ``Simple
Arithmetic'', this chapter aims to show that other formal systems such as
Propositional Logic and Typed Lambda Calculus can be define in the same way,
therefore Phometa can encode them.

For the actual encoding and deeper explanation, please see appendix chapter
\ref{chap:example_propositional_logic} for Propositional Logic and appendix
chapter \ref{chap:example_lambda_calculus} for Typed Lambda Calculus.

\section{Propositional Logic}

Propositional Logic lets us manipulate a statement that has a truth value. For
example ``it is raining'', ``today is Monday'', ``it is raining and today is
Monday'', ``if today is Monday then it is not raining'', and so on.

The most primitive unit is \texttt{<Atom>} where a truth statement cannot be
broken down further e.g. ``it is raining'', ``today is Monday''. Formally
\texttt{<Atom>} can be written in Backus-Naur Form like this

\begin{figure}[H]
\begin{framed}
\begin{lstlisting}[style=bnf]
<Atom> ::= literal comply with regex
             /[a-z][a-zA-Z]*([1-9][0-9]*|'*)/
\end{lstlisting}
\end{framed}
\caption{Backus-Naur Form of \texttt{<Atom>}}
\end{figure}

Basically, \texttt{<Atom>} is just a set of string that the first letter must be
lower case and end with either number or ``\textquotesingle'' sign. For
simplicity will use just $p$, $q$, $r$, $\ldots$ for \texttt{<Atom>}.

Literal is used there is the same as meta variables but only be able to
substitute for itself, e.g. $p$ represent only $p$, not $q$ or $r$ or other
terms of \texttt{<Atom>}.

Next we want to construct more general from of truth statement called
\texttt{<Prop>} which lets \texttt{<Atom>}s combined to each other using
connector such as ``and ($\wedge$)'', ``or ($\vee$)'', ``not ($\neg$)'', ``if
... then ... ($\rightarrow$)'', and ``if and only if ($\leftrightarrow$)''.
\texttt{<Prop>} can be defined as the following

\begin{figure}[H]
\begin{framed}
\begin{lstlisting}[style=bnf]
<Prop> ::= `$\top$' | `$\bot$' | <Atom>
         | `('<Prop> `$\wedge$' <Prop>`)'
         | `('<Prop> `$\vee$' <Prop>`)'
         | `( $\neg$' <Prop>`)'
         | `('<Prop> `$\rightarrow$' <Prop>`)'
         | `('<Prop> `$\leftrightarrow$' <Prop>`)'
         | meta-variables comply with regex
             /[A-Z][a-zA-Z]*([1-9][0-9]*|'*)/
\end{lstlisting}
\end{framed}
\caption{Backus-Naur Form of \texttt{<Prop>}}
\end{figure}

For example ``it is raining and today is Monday'' and ``if today is Monday then
it is not raining'' can be defined as $(r \wedge m)$ and $(m \rightarrow (\neg
r))$ respectively where $r$ is ``it is raining'' and $m$ ``today is Monday''.

\texttt{<Prop>} is similar to \texttt{<Expr>} in the way that it has meta
variables e.g. $(A \vee (\neg A))$ represent any term of \texttt{<Prop>} that
has ``$\vee$'' as the main connector and the right sub-term is the same as the
left sub-term but has ``$\neg$'' cover again.

The difference between \texttt{<Atom>} and meta variables of \texttt{<Prop>} is
that \texttt{<Atom>} represent a specific unbreakable proposition whereas meta
variables of \texttt{<Prop>} represent any term of \texttt{<Prop>}. If we can
proof that $(A \vee (\neg A))$ we can derive $(B \vee (\neg B))$, $((m
\rightarrow (\neg r)) \vee (\neg (m \rightarrow (\neg r))))$ immediately.
However if we can proof $(r \vee (\neg r))$ we might not want to share this to
$(m \vee (\neg m))$ as it is irrelevant. This also explain why \texttt{<Atom>}
use literal and but \texttt{<Prop>} use meta variables.

Now we could start to prove something with \texttt{<Prop>}, however, this is not
so useful as the validity of \texttt{<Prop>} means ``holds in every
circumstance'' but most of \texttt{<Prop>} only holds under curtain assumptions
so we need a way to define set of assumptions (\texttt{<Context>}) and the bigger
(meta) proposition that include assumptions (\texttt{<Judgement>}) which can be
defined as the following

\begin{figure}[H]
\begin{framed}
\begin{lstlisting}[style=bnf]
<Context>     ::= `$\epsilon$'
                | <Context> `,' <Prop>
                | meta-variables comply with regex
                    /[$\Gamma\Delta$]([1-9][0-9]*|'*)/

<Judgement>   ::= <Context> `$\vdash$' <Prop>
\end{lstlisting}
\end{framed}
\caption{Backus-Naur Form of \texttt{<Context>} and \texttt{<Judgement>}}
\end{figure}

For example, ``$\epsilon, p, (q \vee r) \vdash ((p \wedge q) \vee (p \wedge
r))$'' means ``assume that $p$ and $(q \vee r)$ hold then $((p \wedge q) \vee (p
\wedge r))$ also holds''.

We represent \texttt{<Context>} using a list but it is actually a set, hence
ordering and duplication among propositions doesn't matter so we could define
some term to be ``equal by definition'' as the following
\begin{itemize}
\item $\Gamma, A, B$ is definitionally equal to $\Gamma, B, A$
\item $\Gamma, A$ is definitionally equal to $\Gamma, A, A$
\end{itemize}
For example, ``$\epsilon, p, q$'' is definitionally equal to ``$\epsilon, q, p,
q, q, p$'', which in turn, makes ``$\epsilon, p, q \vdash q$'' become
definitionally equal to ``$\epsilon, q, p, q, q, p \vdash q$''.

In order to prove any judgement, we want ability to state that for any
proposition that is in assumptions, it can be conclusion i.e. $\epsilon, A_1,
A_2, \ldots, A_n \vdash A_i$ where $i \in {1..n}$. This is achievable by the
following rule

\begin{figure}[H]
\begin{framed}
\centering
$$
\derivRule{hypothesis}{\Gamma, A \vdash A}{ }
$$
\end{framed}
\caption{Rule \rulename{hypothesis}}
\end{figure}

For example, ``$\epsilon, q, p, q, q, p \vdash r$'' provable using
\rulename{hypothesis}, this because we can change the goal implicitly to
``$\epsilon, p, q \vdash q$'' (by definition) then apply \rulename{hypothesis}
using $\Gamma$ as $\epsilon, p$ and $A$ as $q$. \rulename{hypothesis} doesn't
have any further premise so it is finished.

In the actual encoding of Propositional Logic in Phometa, \rulename{hypothesis}
will automatically search the usable assumption by itself using an additional
feature of Phometa called \emph{cascade premise} which can be seen in appendix
section \ref{sec:hypothesis_rules}.

\newpage

The rest of the rules can be defined as the following

\begin{figure}[H]
\centering
\begin{framed}
$$ \qquad  \qquad  \qquad
\derivRule{top-intro}{\Gamma \vdash \top}{ } \qquad  \qquad  \qquad  \qquad \qquad  \qquad
\derivRule{bottom-elim}{\Gamma \vdash A}{\Gamma \vdash \bot}
$$

$$ \qquad \qquad
\derivRule{and-intro}{\Gamma \vdash (A \wedge B)}{\Gamma \vdash A \\ \Gamma
  \vdash B} \qquad \qquad \qquad \qquad  \qquad
\derivRule{or-intro-left}{\Gamma \vdash (A \vee B)}{\Gamma \vdash A}
$$

$$ \qquad \qquad  \qquad
\derivRule{and-elim-left}{\Gamma \vdash
  A}{\Gamma \vdash (A \wedge B)} \qquad \qquad \qquad  \qquad \qquad
\derivRule{or-intro-right}{\Gamma \vdash (A \vee B)}{\Gamma \vdash B}
$$

$$ \qquad  \qquad  \qquad  \qquad
\derivRule{and-elim-right}{\Gamma \vdash B}{\Gamma \vdash (A \wedge B)} \qquad
\qquad \qquad
\derivRule{or-elim}{\Gamma \vdash C}{\Gamma \vdash (A \vee B) \\
  \Gamma, A \vdash C \\
  \Gamma, B \vdash C}
$$

$$ \qquad \qquad \qquad
\derivRule{not-intro}{\Gamma \vdash (\neg A)}{\Gamma, A \vdash \bot}
  \qquad  \qquad  \qquad  \qquad  \qquad  \qquad  \qquad \qquad
\derivRule{proof-by-contradiction}{\Gamma \vdash A}{\Gamma, (\neg A) \vdash \bot}
$$

$$
\qquad  \qquad
\derivRule{not-elim}{\Gamma \vdash \bot}{\Gamma \vdash (\neg A) \\ \Gamma \vdash
  A}
\qquad  \qquad  \qquad \qquad
\derivRule{iff-intro}{\Gamma \vdash (A \leftrightarrow B)}{\Gamma, A \vdash B
  \\ \Gamma, B \vdash A}
$$

$$
\qquad \qquad \qquad
\derivRule{imply-intro}{\Gamma \vdash (A \rightarrow B)}{\Gamma, A \vdash B}
\qquad \qquad \qquad  \qquad \qquad  \qquad
\derivRule{iff-elim-forward}{\Gamma \vdash B}{\Gamma \vdash (A \leftrightarrow B) \\
  \Gamma \vdash A}
$$

$$ \qquad  \qquad
\derivRule{imply-elim}{\Gamma \vdash B}{\Gamma \vdash (A \rightarrow B) \\
  \Gamma \vdash A}
\qquad  \qquad  \qquad \qquad  \qquad
\derivRule{iff-elim-backward}{\Gamma \vdash A}{\Gamma \vdash (A \leftrightarrow B) \\
  \Gamma \vdash B}
$$
\end{framed}
\caption{Derivation rules of Propositional Logic.}
\end{figure}

Now we are ready to build derivation trees,


\begin{sidewaysfigure}
\centering
$$
\derivTree{or-elim, leftskip=6em}{\epsilon, p, (q \vee r)
                    \vdash ((p \wedge q) \vee (p \wedge r))}
{ \derivTree{hypothesis}{\epsilon, p, (q \vee r) \vdash (q \vee r)}{ }
\\ \\ \\
  \derivTree{or-intro-left}{\epsilon, p, (q \vee r), q
                            \vdash ((p \wedge q) \vee (p \wedge r))}
  { \derivTree{and-intro}{\epsilon, p, (q \vee r), q
                            \vdash (p \wedge q)}
    { \derivTree{hypothesis, rightskip=2em, vdots=2em}{\epsilon, p, (q \vee r), q \vdash p} { }
      \derivTree{hypothesis}{\epsilon, p, (q \vee r), q \vdash q} { }
    }
  }
\\ \\ \\ \\
  \derivTree{or-intro-right}{\epsilon, p, (q \vee r), r
                             \vdash ((p \wedge q) \vee (p \wedge r))}
  { \derivTree{and-intro}{\epsilon, p, (q \vee r), r
                            \vdash (p \wedge r)}
    { \derivTree{hypothesis, rightskip=2em,vdots=2em}{\epsilon, p, (q \vee r), r \vdash p} { }
      \derivTree{hypothesis}{\epsilon, p, (q \vee r), r \vdash r} { }
    }
  }
}
$$
\vspace{7ex}
$$
\derivTree{proof-by-contradiction, leftskip=6em}{\Gamma \vdash (A \vee (\neg A))}
{ \derivTree{not-elim}{\Gamma, (\neg (A \vee (\neg A))) \vdash \bot}
  { \\
    \derivTree{not-intro, leftskip=10em, rightskip=40em, vdots=7em}{\Gamma, (\neg (A \vee (\neg A))) \vdash (\neg A)}
    { \derivTree{not-elim}{\Gamma, (\neg (A \vee (\neg A))), A \vdash \bot}
      { \derivTree{hypothesis}{\Gamma, (\neg (A \vee (\neg A))), A
                               \vdash (\neg (A \vee (\neg A)))} { }
      \\ \\ \\
        \derivTree{or-intro-left}{\Gamma, (\neg (A \vee (\neg A))), A
                                  \vdash (A \vee (\neg A))}
        { \derivTree{hypothesis}{\Gamma, (\neg (A \vee (\neg A))), A
                                  \vdash A} { }
        }
      }
    }
  \\
    \derivTree{proof-by-contradiction}{\Gamma, (\neg (A \vee (\neg A))) \vdash A}
    { \derivTree{not-elim}{\Gamma, (\neg (A \vee (\neg A))), (\neg A)
                           \vdash \bot}
      { \derivTree{hypothesis}{\Gamma, (\neg (A \vee (\neg A))), (\neg A)
                               \vdash (\neg (A \vee (\neg A)))} { }
      \\ \\ \\
        \derivTree{or-intro-right}{\Gamma, (\neg (A \vee (\neg A))), (\neg A)
                                   \vdash (A \vee (\neg A))}
        { \derivTree{hypothesis}{\Gamma, (\neg (A \vee (\neg A))), (\neg A)
                                 \vdash (\neg A)} { }
        }
      }
    }
  }
}
$$
\vspace{3ex}
\caption{Derivation trees show that $\epsilon, p, (q \vee r) \vdash ((p \wedge q) \vee (p \wedge
r))$ and $\Gamma \vdash (A \vee (\neg A))$ are valid \texttt{<Judgement>}s.}
\end{sidewaysfigure}

\section{Typed Lambda Calculus}

Credit: Some of material here modified from lecture note of ``382 --- Type
Systems for Programming Languages'' (third year course), Department of
Computing, Imperial College London. Thank you Dr Steffen van Bakel for this.

Please note that the following formal system is just a faction of what Lambda
Calculus could be. To be precise, this formal system includes $\beta$-reduction
and simply types of Lambda Calculus.

\vspace{-1em}
\subsection{Terms and Variables}

Lambda Calculus stimulates computational model in functional manner. Basically,
it will have a grammar \texttt{<Term>} which can be either a
\texttt{<Variable>}, an anonymous function, an application, or a substitution.
It can be described in Backus-Naur Form like this

\begin{figure}[H]
\begin{framed}
\begin{lstlisting}[style=bnf]
<Variable> ::= meta variable comply with regex /[a-z]([1-9][0-9]*|'*)/

<Term>     ::= <Variable>
             | <Variable> `$\mapsto$' <Term>
             | <Term> `$\cdot$' <Term>
             | <Term> `[' <Term> `/' <Variable> `]'
             | meta variable comply with regex /[L-Z]([1-9][0-9]*|'*)/
\end{lstlisting}
\end{framed}
\caption{Backus-Naur Form of \texttt{<Variable>} and \texttt{<Term>}}
\end{figure}
\vspace{-1em}
The first choice of \texttt{<Term>} represents variable which you can think as a
mark point that other terms can refer to.

The second choice of \texttt{<Term>} represents an anonymous function i.e.\ $(x
\mapsto M)$ means a function that get a variable $x$ and return a term $M$ where
$M$ might contain $x$. From now on, I will call  as ``blinder''

An anonymous function is usually represented by $(\lambda x . M)$ but I use $(x
\mapsto M)$ since it works better with underlines when we implement it in
Phometa.

The third choice of \texttt{<Term>} represents an application i.e. $(M \cdot
N)$ means apply term $M$ to term $N$ where $M$ usually be an anonymous
function\footnote{Or will become anonymous function after evaluate the term.}.

The fourth choice of \texttt{<Term>} represents a substitution i.e.\ ($M [ N / x
]$) means a term $M$ that every occurrence of free variable $x$ will be replaced
by a term $N$.

A variable $x$ is free variable if it doesn't live inside an anonymous function
that has $x$ as blinder\footnote{Blinder is a variable on the left hand side
of $\mapsto$ of an anonymous function}, otherwise, it is bounded variable.

\subsection{Side Conditions}

Sometime we want to build a simple formal system but its dependency relies on
more complex formal system, for example, $\beta$ reduction (will be defined
later) needs to check whether a variable is a free variable of a curtain term or
not, and checking free variable require knowledge of sets. Normally we should
build a formal system regarding to set, then we can build $\beta$ reduction,
however this is overkill as formal system of set is much larger than $\beta$
reduction. To solve this problem, we can build a grammar that construct a
statement that need to be judge by the user as the following

\begin{figure}[H]
\begin{framed}
\begin{lstlisting}[style=bnf]
<SideCondition> ::= <Variable> `$\neq$' <Variable>
                  | <Variable> `$=$' <Variable>
                  | <Variable> `$\in fv($' <Term> `)'
                  | <Variable> `$\notin fv($' <Term> `)'
                  | <Variable> `$\in bv($' <Term> `)'
                  | <Variable> `$\notin bv($' <Term> `)'
                  | <Variable> `$\in fv($' <Context> `)'
                  | <Variable> `$\notin fv($' <Context> `)'
\end{lstlisting}
\end{framed}
\caption{Backus-Naur Form of \texttt{<SideCondition>}}
\end{figure}

Then write a single rule that allow to prove any side condition.

\begin{figure}[H]
\begin{framed}
\centering
$$
\derivRule{side-condition}{SideCondition}{ }
$$
\end{framed}
\caption{Rule \rulename{side-condition}}
\end{figure}

Every time when \texttt{SideCondition} appear in a derivation tree, user needs
extra care and determine whether that side condition holds or not. If it holds
then apply \rulename{side-condition} and it is done. Please note that there is no
mechanism to prevent user to apply \rulename{side-condition} on a false
\texttt{SideCondition} and it will result in inconsistency, in another word, by
using side condition technique, we \emph{trust} user to do the right thing.

\subsection{$\beta$ Reduction}

Since \texttt{<Term>} represents computational model, so it can be
\emph{evaluated}. A step of evaluation in Lambda Calculus is called
\texttt{$\beta$-Reduction} which can be defined as the following


\begin{figure}[H]
\begin{framed}
\begin{lstlisting}[style=bnf]
<$\beta$-Reduction> ::= <Term> `$\rightarrow_{\beta}$' <Term>
\end{lstlisting}
\end{framed}
\caption{Backus-Naur Form of \texttt{<$\beta$-Reduction>}}
\end{figure}

$M \rightarrow_{\beta} N$ means $M$ can reduce to $N$ in one step, the rules to
control this can be defined as the following


\begin{figure}[H]
\begin{framed}
\centering
$$\qquad\qquad
\derivRule{reduction-base}{((x \mapsto M) \cdot N) \rightarrow_{\beta} (M [ N / x ])}{ }
$$

$$\qquad\qquad\qquad\qquad\quad
\derivRule{reduction-app-left}{(M \cdot L) \rightarrow_{\beta} (N \cdot L)}{M \rightarrow_{\beta} N}
\qquad\qquad\qquad\qquad\qquad
\derivRule{reduction-app-right}{(L \cdot M) \rightarrow_{\beta} (L \cdot N)}{M \rightarrow_{\beta} N}
$$
$$\qquad\qquad
\derivRule{reduction-abs}{(x \mapsto M) \rightarrow_{\beta} (x \mapsto N)}{M \rightarrow_{\beta} N}
$$
$$\qquad\qquad\qquad\qquad
\derivRule{reduction-subst-left}{(M [ L / x ]) \rightarrow_{\beta} (N [ L / x ])}{M \rightarrow_{\beta} N}
$$
$$\qquad\qquad\qquad\qquad
\derivRule{reduction-subst-right}{(L [ M / x ]) \rightarrow_{\beta} (L [ N / x ])}{M \rightarrow_{\beta} N}
$$

$$\qquad\qquad\qquad\quad
\derivRule{subst-same-var}{(x [ N / x ]) \rightarrow_{\beta} N}{ }
\qquad\qquad\quad
\derivRule{subst-app}{((P \cdot Q) [ N / x ]) \rightarrow_{\beta} ((P [ N / x ]) \cdot (Q [ N / x ]))}{ }
$$
$$\qquad\qquad\qquad
\derivRule{subst-diff-var}{(y [ N / x ]) \rightarrow_{\beta} y}{y \neq x}
\qquad\qquad\qquad
\derivRule{subst-abs}{((y \mapsto M) [ N / x ]) \rightarrow_{\beta} (y \mapsto (M [ N / x ]))}{y \notin fv( N ) \\ y \neq x}
$$

$$\qquad\qquad\qquad
\derivRule{subst-abs-renaming}{((y \mapsto M) [ N / x ]) \rightarrow_{\beta} ((z
  \mapsto (M [ z / y ])) [ N / x ])}{y \in fv( N ) \\ y \neq x}
$$
$$\qquad\qquad\qquad
\derivRule{subst-abs-discard}{((x \mapsto M) [ N / x ]) \rightarrow_{\beta} (x
  \mapsto M)}{ }
$$
\end{framed}
\caption{Rules related to \texttt{<$\beta$-Reduction>}}
\end{figure}

There is also \texttt{$\beta$-ManyReduction} which reflexive transitive closure
of \texttt{$\beta$-Reduction}.

\begin{figure}[H]
\begin{framed}
\begin{lstlisting}[style=bnf]
<$\beta$-ManyReduction> ::= <Term> `$\rightarrow^*_{\beta}$' <Term>
\end{lstlisting}
$$\qquad\qquad\qquad\qquad\qquad
\derivRule{many-reduction-base}{M \rightarrow^*_{\beta} M}{ }
\qquad\qquad\qquad\qquad\qquad\qquad
\derivRule{many-reduction-next}{M \rightarrow^*_{\beta} N}{M \rightarrow_{\beta}
  L \\ L \rightarrow^*_{\beta} N}
$$
\end{framed}
\caption{Backus-Naur Form of \texttt{<$\beta$-ManyReduction>} and its rules.}
\end{figure}

Now we are ready to build a derivation tree,

\begin{figure}[H]
\centering
$$
\derivTree{many-reduction-next, leftskip=9em}{((x \mapsto x) \cdot (y \mapsto y))
  \rightarrow^*_{\beta} (y \mapsto y)}
{ \derivTree{reduction-base}{((x \mapsto x) \cdot (y \mapsto y))
  \rightarrow_{\beta} (x [ (y \mapsto y) / x ])} { }
\\ \\ \\
  \derivTree{many-reduction-next, leftskip=20em, vdots=3em}{(x [ (y \mapsto y) / x ])
  \rightarrow^*_{\beta} (y \mapsto y)}
  { \derivTree{subst-same-var}{(x [ (y \mapsto y) / x ])
    \rightarrow_{\beta} (y \mapsto y)} { }
  \\ \\ \\ \\ \\
    \derivTree{many-reduction-base}{(y \mapsto y)
    \rightarrow^*_{\beta} (y \mapsto y)}
  { }
  }
}
$$
\caption{Derivation tree shows that $((x \mapsto x) \cdot (y \mapsto y))
  \rightarrow^*_{\beta} (y \mapsto y)$ \\ is valid a \texttt{<$\beta$-ManyReduction>}.}
\end{figure}

\subsection{Simply Types}

Some terms can be reduced infinitely many times e.g. \\
$((x \mapsto (x \cdot x)) \cdot (x \mapsto (x \cdot x))) \rightarrow_{\beta}$
$((x \cdot x) [ (x \mapsto (x \cdot x)) / x ]) \rightarrow_{\beta}$ \\
$((x [ (x \mapsto (x \cdot x)) / x ]) \cdot (x [ (x \mapsto (x \cdot x)) / x
])) \rightarrow_{\beta}$
$((x \mapsto (x \cdot x)) \cdot (x [ (x \mapsto (x \cdot x)) / x
])) \rightarrow_{\beta}$ \\
$((x \mapsto (x \cdot x)) \cdot (x \mapsto (x \cdot x))) \rightarrow_{\beta} \cdots$

This is not desirable as it implies a program that will never terminate. To
solve this problem we can try to find a type of a term, if found, we know that
the term as finite number of reductions. \texttt{<Type>} can be defined as the following


\begin{figure}[H]
\begin{framed}
\begin{lstlisting}[style=bnf]
<Type> ::= <Type> `$\rightarrow$' <Type>
         | meta variable comply with regex /[A-K]([1-9][0-9]*|'*)/
         | literal comply with regex /[$\alpha$-$\omega$]([1-9][0-9]*|'*)/
\end{lstlisting}
\end{framed}
\caption{Backus-Naur Form of \texttt{<Type>}}
\end{figure}

\texttt{<Type>} can be instantiated as
\begin{itemize}
\item meta variable --- for generic type which can be used in another place.
\item literal --- for constant type similarly to Int, Bool, Char, etc.
\item $(A \rightarrow B)$ --- a type of anonymous function that receive input
  of type $A$ and return output of type $B$.
\end{itemize}

To state that a term has the particular type we can use \texttt{<Statement>},
and in order to prove a that \texttt{<Statement>} holds under certain
assumptions, we need \texttt{<Context>} and \texttt{<Judgement>}. These three
grammars can be defined as the following

\begin{figure}[H]
\begin{framed}
\begin{lstlisting}[style=bnf]
<Statement>   ::= <Term> `:' <Type>

<Context>     ::= `$\epsilon$'
                | <Context> `,' <Prop>
                | meta-variables comply with regex
                    /[$\Gamma\Delta$]([1-9][0-9]*|'*)/

<Judgement>   ::= <Context> `$\vdash$' <Statement>
\end{lstlisting}
\end{framed}
\caption{Backus-Naur Form of \texttt{<Statemeny>}, \texttt{<Context>}, and \texttt{<Judgement>}}
\end{figure}

The rules regarding to type can be defined as the following,


\begin{figure}[H]
\begin{framed}
$$
\derivRule{assumption}{\Gamma, (M : A) \vdash (M : A)}{ }
$$

$$
\derivRule{arrow-intro}{\Gamma \vdash ((x \mapsto M) : (A \rightarrow B))}
   {\Gamma, (x : A) \vdash (M : B) \\ x \notin fv( \Gamma )}
$$

$$
\derivRule{arrow-elim}{\Gamma \vdash ((M \cdot N) : B)}
   {\Gamma \vdash (M : (A $$\rightarrow$$ B)) \\ \Gamma \vdash (N : A)}
$$
\end{framed}
\caption{Rules related to \texttt{<Type>}.}
\end{figure}

The \rulename{assumption} is very similar to \rulename{hypothesis} in
Propositional Logic. It also has cascade premise and context manipulation.

Now we are ready to build a derivation tree,

\begin{sidewaysfigure}
\centering
$$
\derivTree{arrow-intro, leftskip=5em}{\epsilon \vdash ((f \mapsto (x \mapsto (f
  \cdot x))) : ((A \rightarrow A) \rightarrow (A \rightarrow A)))}
{
  \derivTree{arrow-intro}{\epsilon, (f : (A \rightarrow A)) \vdash ((x \mapsto
    (f \cdot x)) : (A \rightarrow A))}
  { \derivTree{arrow-elim, rightskip=25em, vdots=3em}{\epsilon, (f : (A \rightarrow A)), (x : A) \vdash (
    (f \cdot x) : A)}
    { \derivTree{assumption}{\epsilon, (f : (A \rightarrow A)), (x : A) \vdash (
    f : (A \rightarrow A))}{ }
      \\ \\ \\ \\ \\
      \derivTree{assumption}{\epsilon, (f : (A \rightarrow A)), (x : A) \vdash (
    x : A)}{ }
    }
  \\ \\ \\ \\
    \derivTree{side-condition}{f \notin fv( \epsilon, (f : (A \rightarrow A)) )}{ }
  }
\\ \\ \\ \\
  \derivTree{side-condition}{f \notin fv( \epsilon )}{ }
}
$$
\caption{Derivation tree shows that $\epsilon \vdash ((f \mapsto (x \mapsto (f
  \cdot x))) : ((A \rightarrow A) \rightarrow (A \rightarrow A)))$ is valid a
  \texttt{<Judgement>}.}
\end{sidewaysfigure}


\end{document}
