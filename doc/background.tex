\documentclass[master.tex]{subfiles}
\begin{document}

\chapter{Background}
\label{chap:background}

In this chapter, we will go thought some required materials needed for later
chapters. These can be linked together by an example of Simple Arithmetic
explained below.

\hspace{1ex}

\section{Formal Systems}

A formal system is any well-defined system of abstract thought based on
a mathematical model\supercite{formal-system-wiki}. Each formal system has a
formal language composed of primitive symbols\footnote{Phometa will assume that
  primitive symbols are any Unicode character.} acted by certain
formation\supercite{formal-system-britannica}.

Informally, it is an abstract system that has precise structures and can be
reasoned about. For example, numbers (base 10) and their arithmetic (using $+$
and $\times$) could form a formal system. This is because every term (e.g. $5$,
$(3 + 1)$, $(3 \times 4)$) has explicit structure and we can argue something
like ``\emph{does $12$ equal to $(3 \times 4)$}'' or ``\emph{for any integers
  $a$ and $b$, $(a + b)$ is equal to $(b + a)$}''.

\hspace{1ex}

\section{Backus-Naur Form}

Backus-Naur Form (BNF) is a way to construct a term, for example, grammars of
the formal system above can be defined as the following

\hspace{1ex}

\begin{figure}[H]
\begin{framed}
\begin{lstlisting}[style=bnf]
<Digit>  ::= `0' | `1' | `2' | `3' | `4' | `5' | `6' | `7' | `8' | `9'

<Number> ::= <Digit> | <Number> <Digit>

<Expr>   ::= <Number>
           | `('<Expr> `+' <Expr>`)'
           | `('<Expr> `$\times$' <Expr>`)'

<Equation> ::= <Expr> `=' <Expr>
\end{lstlisting}
\end{framed}
\caption{Backus-Naur Form of Simple Arithmetic}
\label{fig:background-bnf}
\end{figure}

\hspace{2ex}

\begin{itemize}
\item A term of \texttt{<Digit>} can be either 0 or 1 or 2 or $\ldots$ or 9 and nothing else.

\begin{figure}[H]
\begin{framed}
  \centering $2$ is \texttt{<Digit>} ($3^{rd}$ choice)
\end{framed}
    \caption{This diagram explains why $2$ is a term of \texttt{<Digit>}}
\label{fig:background-digit}
\end{figure}

\hspace{1ex}

\item A term of \texttt{<Number>} can be either
\begin{itemize}
\item \texttt{<Digit>}
\item another \texttt{<Number>} concatenated with \texttt{<Digit>}
\end{itemize}

\begin{figure}[H]
\begin{framed}
  \centering \Tree[.{$250$ is \texttt{<Number>} ($2^{nd}$ choice)} [.{$25$ is \texttt{<Number>} ($2^{nd}$ choice)}
  [.{$2$ is \texttt{<Number>} ($1^{st}$ choice)} {$2$ is \texttt{<Digit>} ($3^{rd}$ choice)} ] {$5$ is
    \texttt{<Digit>} ($6^{th}$ choice)} ] {$0$ is \texttt{<Digit>} ($1^{st}$ choice)} ]
\end{framed}
    \caption{This diagram explains why $250$ is a term of \texttt{<Number>}}
\label{fig:background-number}
\end{figure}

\item A term of \texttt{<Expr>} can be either
\begin{itemize}
\item \texttt{<Number>}
\item two \texttt{<Expr>}s concatenated using `$($' `$+$' `$)$'
\item two \texttt{<Expr>}s concatenated using `$($' `$\times$' `$)$'
\end{itemize}

Please note that we need brackets around `$+$' and `$\times$' to avoid
ambiguity. If we don't have these brackets, $3 + 4 + 5$ could be interpreted as
either $(3 + 4) + 5$ or $3 + (4 + 5)$ which is not precise. Moreover, $12 + 0
\times 6$ will be interpreted as $12 + (0 \times 6)$ due to priority of $\times$
over $+$ and it is impossible to encode something like $(12 + 0) \times 6$.

\begin{figure}[H]
\begin{framed}
    \centering
\Tree[.{$((3 + 4) + 5)$ is \texttt{<Expr>} ($2^{nd}$ choice)}
       [.{$(3 + 4)$ is \texttt{<Expr>} ($2^{nd}$ choice)}
         [.{$3$ is \texttt{<Expr>} ($1^{st}$ choice)} [.{$3$ is
           \texttt{<Number>}} {$3$ is \texttt{<Digit>}} ] ]
         [.{$4$ is \texttt{<Expr>} ($1^{st}$ choice)} [.{$4$ is
           \texttt{<Number>}} {$4$ is \texttt{<Digit>}} ] ] ]
       [.{$5$ is \texttt{<Expr>} ($1^{st}$ choice)} [.{$5$ is \texttt{<Number>}} {$5$ is \texttt{<Digit>}} ] ] ]
\end{framed}
    \caption{This diagram explains why $((3 + 4) + 5)$ is a term of \texttt{<Expr>}}
\label{fig:background-expr}
\end{figure}

\begin{figure}[H]
\begin{framed}
    \centering
\Tree[.{$(12 + (0 \times 6))$ is \texttt{<Expr>} ($2^{nd}$ choice)}
       [.{$12$ is \texttt{<Expr>} ($1^{st}$ choice)}
         [.{$12$ is \texttt{<Number>}}
           [.{$1$ is \texttt{<Number>}} {$1$ is \texttt{<Digit>}} ]
           {$2$ is \texttt{<Digit>}} ] ]
       [.{$(0 \times 6)$ is \texttt{<Expr>} ($3^{rd}$ choice)}
         [.{$0$ is \texttt{<Expr>} ($1^{st}$ choice)} [.{$0$ is \texttt{<Number>}} {$0$ is \texttt{<Digit>}} ] ]
         [.{$6$ is \texttt{<Expr>} ($1^{st}$ choice)} [.{$6$ is \texttt{<Number>}} {$6$ is \texttt{<Digit>}} ] ]]]
\end{framed}
    \caption{This diagram explains why $(12 + (0 \times 6))$ is a term of \texttt{<Expr>}}
\label{fig:background-expr-2}
\end{figure}


\item A term of \texttt{<Equation>} can be only two \texttt{<Expr>}s concatenated using
`$=$'

\begin{figure}[H]
\begin{framed}
    \centering
\Tree[.{$(5 + 7) = 12$ is \texttt{<Equation>} ($1^{st}$ choice)}
       [.{$(5 + 7)$ is \texttt{<Expr>} ($2^{nd}$ choice)}
         [.{$5$ is \texttt{<Expr>} ($1^{st}$ choice)} [.{$5$ is \texttt{<Number>}} {$5$ is
           \texttt{<Digit>}} ] ]
         [.{$7$ is \texttt{<Expr>} ($1^{st}$ choice)} [.{$7$ is \texttt{<Number>}} {$7$ is
           \texttt{<Digit>}} ] ] ]
      [.{$12$ is \texttt{<Expr>} ($1^{st}$ choice)}
         [.{$12$ is \texttt{<Number>}} [.{$1$ is \texttt{<Number>}} {$1$ is \texttt{<Digit>}} ]
      {$2$ is \texttt{<Digit>}} ] ] ]
\end{framed}
    \caption{This diagram explains why $(5 + 7) = 12$ is a term of \texttt{<Equation>}}
\label{fig:background-equation}
\end{figure}

\begin{figure}[H]
\begin{framed}
    \centering
\Tree[.{$(2 \times 3) = 5$ is \texttt{<Equation>} ($1^{st}$ choice)}
       [.{$(2 \times 3)$ is \texttt{<Expr>} ($3^{rd}$ choice)}
         [.{$2$ is \texttt{<Expr>} ($1^{st}$ choice)} [.{$2$ is \texttt{<Number>}} {$2$ is
           \texttt{<Digit>}} ] ]
         [.{$3$ is \texttt{<Expr>} ($1^{st}$ choice)} [.{$3$ is \texttt{<Number>}} {$3$ is
           \texttt{<Digit>}} ] ] ]
       [.{$5$ is \texttt{<Expr>} ($1^{st}$ choice)} [.{$5$ is \texttt{<Number>}} {$5$ is
         \texttt{<Digit>}} ] ] ]
\end{framed}
    \caption{This diagram explains why $(2 \times 3) = 5$ is a term of
      \texttt{<Equation>}. Please note that this construction is purely
      syntactic so wrong equation is acceptable.}
\label{fig:background-equation-2}
\end{figure}

\end{itemize}

\hspace{1ex}

\section{Meta Variables and Pattern Matching}

\emph{Meta variables} are arbitrary sub-terms embedded inside a root term. For
example, an \texttt{<Expr>} $(x + y)$ represents two arbitrary \texttt{<Expr>}
joined by `$+$'.

But if we have an \texttt{<Equation>} $(x + 7) = 12$, shouldn't $x$ be an unknown
variable that needed to be solve rather than being arbitrary \texttt{<Expr>} ?
Well, $x$ still represents arbitrary \texttt{<Expr>} but in order make this
equation hold, $x$ must be $5$. Hence ``\emph{variable needed to be solve}'' is
just a special form of ``\emph{variable as arbitrary term}''.

Meta variables help us to represent statements in a more general manner. For
example, ``\emph{the same expressions added together is the same as 2 times that
  expression}'' could be represented by $(x + x) = (2 \times x)$ rather than $(0
+ 0) = (2 \times 0)$ and $(1 + 1) = (2 \times 1)$ and $(2 + 2) = (2 \times 2)$
and so on.

But if we know that $(x + x) = (2 \times x)$, how could we derive its instance
e.g. $(1 + 1) = (2 \times 1)$ or even $(y \times z) + (y \times z) = (2 \times
(y \times z))$ ? The solution for this is to use \emph{Pattern Matching} which
is an algorithm to substitute pattern's meta variables into more specific form,
in order to make pattern identical to target, for example
\begin{itemize}
\item $(x + x) = (2 \times x)$ is pattern matchable with $(1 + 1) = (2 \times 1)$ by
  substitute $x$ with$1$
\item $(x + x) = (2 \times x)$ is pattern matchable with $(y \times z) + (y
  \times z) = (2 \times (y \times z))$ by substitute $x$ with $(y \times z)$
\item $(x + x) = (2 \times x)$ is \emph{not} pattern matchable with $(1 + 1) = 2$, if
  we try to substitute $x$ with $1$ we would get $(1 + 1) = (2 \times 1)$ which is not
  identical to $(1 + 1) = 2$
\item $(1 + 1) = (2 \times 1)$ is \emph{not} pattern matchable with $(x + x) =
  (2 \times x)$, because pattern $(1 + 1) = (2 \times 1)$ doesn't have any meta
  variable and it is not identical to $(x + x) = (2 \times x)$. This show that
  pattern matching doesn't generally holds in opposite direction
\item $(x + x) = (2 \times x)$ is pattern matchable to itself by substitute $x$ with $x$
\end{itemize}

If pattern matching is successful then the target is an instance of the pattern.

\section{Derivation of Formal Systems}

So far, we construct any term based on Backus-Naur Form, this doesn't prevent
invalid term, for example, $(2 \times 3) = 5$ is perfectly a term of
\texttt{<Equation>}. Thus, we need some mechanism to verify a term i.e.
\emph{prove} that the particular term holds. One way to deal with this is to use
a derivation system. First, we have a set of derivation rules that has format as
the following

\begin{figure}[H]
\centering
$$
\qquad \qquad \derivRule{rule-name}{Conclusion}{Premise_1 \\ Premise_2 \\ Premise_3 \\ \ldots \\
  Premise_n}
$$
\caption{Structure of derivation rule.}
\end{figure}
\newpage

This says that if we know that $Premise_1$ and $Premise_2$ and $Premise_3$ and
$\ldots$ and $Premise_n$ hold then $Conclusion$ holds. In another word, if we
want to prove $Conclusion$ then we can use this derivation and then proof its
premises.

Derivation rules of the current example formal system could be shown as the
following

\begin{figure}[H]
\centering
$$
\derivRule{eq-refl}{x = x}{ } \qquad  \qquad  \qquad
\derivRule{eq-symm}{x = y}{y = x} \qquad  \qquad  \qquad
\derivRule{eq-tran}{x = y}{x = z \\ z = y}
$$

$$
\derivRule{add-intro}{(u + v) = (x + y)}{u = x \\ v = y} \qquad  \qquad \qquad \qquad
\derivRule{mult-intro}{(u \times v) = (x \times y)}{u = x \\ v = y}
$$

$$
\derivRule{add-assoc}{((x + y) + z) = (x + (y + z))}{ } \qquad  \qquad  \qquad \qquad
\derivRule{mult-assoc}{((x \times y) \times z) = (x \times (y \times z))}{ }
$$

$$
\derivRule{add-comm}{(x + y) = (y + x)}{ } \qquad  \qquad  \qquad \qquad
\derivRule{mult-comm}{(x \times y) = (y \times x)}{  }
$$

$$
\derivRule{dist-left} {(x \times (y + z)) = ((x \times y) + (x
  \times z))}{ } \qquad \qquad  \qquad
\derivRule{dist-right}{((x + y) \times z) = ((x \times z) + (y
  \times z))}{ }
$$
\caption{Derivation rules of Simple Arithmetic (not exhaustive, due to limited
  space).}
\label{fig:background-derivation-rules}
\end{figure}

In order to use a derivation rule, first the conclusion of the rule is pattern
matched against the current goal. If it is pattern matchable then meta variables
in premises are substituted respect to the pattern matching\footnote{If some
  meta variables of premises doesn't exist in substitution list then we are free
  to substitute by anything e.g.\ in \rulename{eq-tran}, we can replace $z$ with
  arbitrary \texttt{<Expr>}, this free substitution gives a better control over
  premises, which is important when we apply rules to sub-goals.}. These
substituted premises will become next goals that we need to prove.

For example if we want to prove $((3 + 4) * 5) = ((4 + 3) * 5)$ we could use
rule \rulename{mult-intro} to prove it since $(u \times v) = (x \times y)$ is
pattern matchable with $((3 + 4) * 5) = ((4 + 3) * 5)$ by substitute $u$ with $(3 +
4)$, $v$ with $5$, $x$ with $(4 + 3)$, and $y$ with $5$. Then premises $u = x$
and $v = y$ are substituted and become $(3 + 4) = (4 + 3)$ and $5 = 5$
respectively. Therefore, $((3 + 4) * 5) = ((4 + 3) * 5)$ can be proven by
\rulename{mult-intro} and produce another two sub-goals which are $(3 + 4) =
(4 + 3)$ and $5 = 5$. This can be shown as instance of \rulename{mult-intro}
as the following

\begin{figure}[H]
\centering
$$ \derivTree{mult-intro}{((3 + 4) * 5) = ((4 + 3) * 5)}{(3 + 4) = (4 + 3) \\ 5 = 5} $$
\caption{Example of instance of derivation rule.}
\end{figure}

For the remaining, we could prove $(3 + 4) = (4 + 3)$ using
\rulename{add-comm} because $(x + y) = (y + x)$ is pattern matchable with $(3
+ 4) = (4 + 3)$, \rulename{add-comm} doesn't have any premises hence there are
no further sub-goal. For $5 = 5$ we could use \rulename{eq-refl}, this also
deosn't produce further sub-goal so the entire proof is complete. We can the
write the entire proof using \emph{derivation tree} as the following

\begin{figure}[H]
\centering
$$ \derivTree{mult-intro}{((3 + 4) * 5) = ((4 + 3) * 5)}
     {\derivTree{add-comm}{(3 + 4) = (4 + 3)}{ } \\ \\ \\
      \derivTree{eq-refl}{5 = 5}{ }} $$
\caption{Example of derivation tree.}
\label{fig:background-derivation-tree-1}
\end{figure}

Some rules in figure \ref{fig:background-derivation-rules} don't have any
premises. This is necessary, otherwise, applying a rule always generates further
sub goals and the process would never terminate. These rules can be seen as
\emph{axioms} which is a term that is valid by assumption i.e.\ so need to prove
such a term.

For better understanding about a derivation system, here is a more complex
derivation tree which proves $(((w \times x) + (w \times y)) \times z) = (w
\times ((x \times z) + (y \times z)))$. The reader is encouraged to explore that
why this derivation tree is correct.

\vspace{-1em}

\begin{figure}[H]
\centering
$$
\derivTree{eq-tran}{(((w \times x) + (w \times y)) \times z) =
                    (w \times ((x \times z) + (y \times z)))}
{ \derivTree{mult-intro}{(((w \times x) + (w \times y)) \times z) =
                         ((w \times (x + y)) \times z)}
  { \derivTree{eq-symm}{((w \times x) + (w \times y)) =
                         (w \times (x + y))}
    { \derivTree{dist-left}{(w \times (x + y)) = ((w \times x) + (w \times y))} { }
    }
   \\ \\ \\
    { \derivTree{eq-refl}{z = z}{ }
    }
  }
\\ \\ \\ \\
  \derivTree{eq-tran, leftskip=30em,rightskip=5em,vdots=6em}{((w \times (x + y)) \times z) =
                      (w \times ((x \times z) + (y \times z)))}
  { \derivTree{mult-assoc, leftskip=10em,rightskip=20em,vdots=4em}{((w \times (x + y)) \times z) =
                           (w \times ((x + y) \times z))} { }
  \\ \\ \\
    \derivTree{mult-intro}{(w \times ((x + y) \times z)) =
                      (w \times ((x \times z) + (y \times z)))}
    { \derivTree{eq-refl}{w = w}{ }
    \\ \\ \\
      \derivTree{dist-right}{((x + y) \times z) = ((x \times z) + (y \times z))}
      { }
    }
  }
}
$$
\caption{Example of more complex derivation tree.}
\label{fig:background-derivation-tree-2}
\end{figure}

\end{document}