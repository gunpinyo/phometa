\documentclass[master.tex]{subfiles}
\begin{document}

\chapter{Conclusion}

At the end of this project. Phometa has been designed and been implemented up to
the level that is ready to use by anybody with a decent standard library and
tutorial. This, in tern, satisfies all of objective stated in introduction
chapter. In addition, I also believe that Phometa on this state is a potential
replacement for derivation-tree's manually-drawing so people don't have to
suffer from it tedious process and error prone anymore.

\section{Lesson Learnt}

Time management for research project is one of many thing that I have learnt
during this project. I learnt that tasks are always take time twice or thrice
longer than expectation so it is vital to spare plenty of time before the
deadline. More importantly, I learnt that better idea of feature always come
after we start to implement something. It is quite hard to decide whether
Phometa should include some certain feature or not. It is about a tread off
between usefulness of the feature and the risk of the project being not finish
in time. This kind of features usually came near the end of implementation where
I knew exactly what Phometa should be. This is bad because if I accepted the
feature, this would take sometime to implement and edit related part of this
report, which in tern, would impact the entire schedule of the plan. So I
usually take it as future work as described in next section.

I also learnt to believe in myself being capable to building something I dream
of. Formal proof always be my favourite topic since I studied Logic in the first
year. One day, I was drawing a derivation tree for a coursework, I had the idea
of this project. At the first time it seemed too scary because it is about
building a proof assistant from scratch, however after evaluated proof of
concept, it turned out to be feasible. So I decide to start it and approached my
supervisor.

Most importantly, I learnt many thing regarding to formal proof from this
project which is relevant to the topic that I want to do for PhD (Dependent Type
Theory). This gave me more familiarity and confident in that field. Oppositely,
curiosity on the field motivated me to work on this project better since I know
that this kind of knowledge gained during the project will be useful later for
sure.

\section{Future Works}

Although Phometa has been designed and implemented up to satisfactory level.
There are still plenty of room for improvement. For example,
\begin{itemize}
\item \textbf{Making Grammars extensible} --- Grammars should be able to extend another
  grammar in similar manner to how class in Object Oriented Programming extend
  other classes, for example, grammar of proposition in First Order Logic could
  extend grammar of proposition in Propositional Logic. This allows polymorphism
  where a term of extending grammar could be proven using rules or lemmas
  related based grammar.
\item \textbf{Plain-text as alternative term input method} --- When constructing a term,
  user should be able to write some term in plain-text and Phometa will try to
  parse it according to formats of that grammar.
\item \textbf{Making Theorems construction become more automatics} --- When constructing a
  theorem, Phometa should be able to guess what rules or lemmas should be
  applied next (similar to Isabelle).
\item \textbf{Importation between Modules} --- Modules should be able to import some
  nodes from other modules, this improve scalability since a formal system can
  be expand across multiple modules.
\item \textbf{Exportation to \LaTeX} --- Grammars, Rules, and Theorem should be
  exportable to \LaTeX\ source code. So it can be used further other document
  such as report or presentation.
\item \textbf{Repository Verification on Loading} --- Creating a function that check
  whether the repository is in valid state or not. Normally, it is not necessary
  as Phometa will always be in consistent state, but when the you want use other
  people formal system by obtaining \texttt{repository.json}, it is not a
  grantee that someone will not modify that repository directly. So it is better
  to have mechanism to check consistency of the repository anyway.
\end{itemize}

There are also other minor improvements such as adding new formal systems to
standard library, making notification message become more informative, deploy
phometa to a proper server, making user interface become more mobile friendly,
and making a theme and keyboard shortcut become configurable.

\end{document}