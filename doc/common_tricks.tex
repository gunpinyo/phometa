\documentclass[master.tex]{subfiles}

\chapter{Common tricks, that user should know}

\section{using condition during define grammar}

Some of logical framework, also have a condition that if it doesn't hold then the particular term in not well-from.

Unfortunately, you can't encode general condition during grammar construction since they are created structurally. However you can solve this using \kRule.

Say that whenever you want to a rule involve term \pvar{t} with (conditional related) grammar \pgmr{g} you need to state

\kPremise \pvar{t} \pgmr{g}

inside that rule as well. Then using other rules which control condition of grammar \pgmr{g} to prove that \pvar{t} holds. Obviously if that term is not well-form, you can't prove that term.

Please note that you only need to do this at the point that \pvar{t} is introduce (or reconstruct). If \pvar{t} already exists in one of your premises, you know that this premise also contain the prove that \pvar{t} holds.

TODO: give an example involve this situation.

%%% Local Variables:
%%% mode: latex
%%% TeX-master: "master"
%%% End:
