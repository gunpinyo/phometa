\documentclass[master.tex]{subfiles}
\begin{document}

\chapter{Installation and Startup}
\label{chap:installation_startup}

You can download the complied version of Phometa at

{\centering\url{https://github.com/gunpinyo/phometa/raw/master/build/phometa.tar.gz}}

Once you unzip this file, you see several files as the following

\begin{itemize}
\item \texttt{\textbf{phometa.html}} --- the main file containing Phometa
  web-based application.
\item \texttt{\textbf{style.css \& style.css.map}} --- control the layout and
  theme of \texttt{\textbf{phometa.html}}
\item \texttt{\textbf{naive.js}} --- JavaScript code that will be injected to
  \texttt{\textbf{phometa.html}}
\item \texttt{\textbf{repository.json}} --- contain proofs repository, can be
  loaded or saved by \texttt{\textbf{phometa.html}}
\item \texttt{\textbf{logo.png}} --- image that will be used as favicon in
  \texttt{\textbf{phometa.html}}
\item \texttt{\textbf{phometa-server.py}} --- python script that can be executed
  as local server.
\item \texttt{\textbf{phometa-doc.pdf}} --- this report, included here as the
  tutorial and reference.
\end{itemize}

You can start Phometa server by execute

\texttt{./phometa-server.py 8080}

where \texttt{8080} is port number, you can change this to another port number
if you like. Please note that Python is required for this server.

Then open your favourite web-browser\footnote{but Google Chrome is recommended}
and enter

\texttt{http://localhost:8080/phometa.html}

The program will look like figure \ref{fig:specification-phometa-home-window}

\end{document}
