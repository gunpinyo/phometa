\documentclass[master.tex]{subfiles}
\begin{document}

\chapter{Evaluation and Conclusion}
\label{chap:evaluation_conclusion}

\vspace{-1em}

\section{Users Feedback --- discuss with friends}

On the $25^{th}$ of May 2016, it was the first day of a project fair where
students can demonstrate their work to other students to get a feedback so I
went there and discuss about my project. At this stage, the implementation was
finished with Simple Arithmetic and Propositional Logic included in the standard
library.

I started showing my project by explaining about Phometa background and Simple
Arithmetic using chapter \ref{chap:background} and
\ref{chap:example_simple_arithmetic} on this report. Then I asked them do to
exercises on chapter \ref{chap:example_simple_arithmetic} by having me as
helper. All of them understood Phometa and was able to prove a theorem. Finally,
I asked them to try Propositional Logic, some of them showed interest but not
all of them wanted to.

From my observation, all of them were comfortable to prove by clicking options
from keymap pane rather than using keyboard shortcut. They also forgot to use
searching pattern to select options faster.

There were a few parts of user interface that were not trivial enough, they
needed to ask me what to do next, this should be fine if user had time to read
the whole tutorial.

On the bright side, most of them said that they really liked the way that
underlines was use to group sub-terms rather than brackets (although they needed
some time to familiar with it), they also said that the proofs are quite easy to
read and they will benefit newcomers.

There were several improvements that they suggested. Some suggestions
were easy to change (e.g.\ theorems should state its goal on header as well) so
I changed it already. Some of other suggestions were quite big (e.g.\ make it
mobile friendly and have a proper server) which can be considered as future
works. We also managed to find some bugs\footnote{These bug are
  related Html and CSS rendering i.e.\ they are not related to Phometa internal.} that I never found before,
this gave me an opportunity to fix it in time.

\section{Users Feedback --- discuss with junior students}

The project fair happened again on the $1^{st}$ of June. This time, I gave a
demonstration to the first and second year students who study here.

The process was the same as last week. I started the demonstration by giving a
Phometa overview using background and Simple Arithmetic chapters and let them
try natural deduction in Propositional Logic. At first, they usually
made mistakes especially on how to construct a term, for example,
\begin{itemize}
\item One student wrote a whole propositional formula in plain-text because he
  thought that is how Phometa accept a term.
\item A few student students forgot to hit \pkbd{Return} when finish entering
  a meta variable. Actually, this should be fine but I disabled such a feature
  because it has a problem with Phometa's internal mode.
\item One group of students accidentally clicked theorem reset button
  (\resetButton) because they thought that was an undo button. This reset button
  there was quite dangerous as it was nearly equivalent to delete button.
\end{itemize}
These problem occur only for the first time, and once they understood more of
the tool, they could use it easily. In fact, the tutorial that I have written
already covered most of these mistake, but I didn't ask students to read though
the entire tutorial due to time constraints.

On the other hand, I was surprised at how fast first year students\footnote{Most
  of them were tutees in Personal Mathematics Tutorial (PMT) group that I took a
  role as Undergraduate Teaching Assistant (UTA) in this academic year.} can
construct Gentzen style's natural deduction in Phometa given that they know how
to do Fitch style's natural deduction in Pandora represented in the first year
Logic course. Moreover, one of second students told me that he didn't
particularly like derivation trees\footnote{Second year students had studied
  derivation tree in the class already.} but Phometa managed to make him enjoy
constructing this derivation tree.

In conclusion, everybody who tries Phometa in the project fair was able to use
Phometa properly when they got a proper guidance. Some students really enjoyed
proving things in Phometa and some students believe that Phometa could have
helped them during their coursework that require constructing derivation trees
in the past.

\section{Strengths}

\begin{itemize}
\item Phometa specification itself is more powerful than traditional derivation
  systems because it has extra features such as cascade premise and
  meta-reduction. Thus, it is able to support more formal systems than
  traditional one.
\item It has a less steep learning curve than mainstream proof assistants
  because the specification is small enough for a user to learn in a short time
  and all of components are diagram based which is easier to understand than
  sequences of characters.
\item If a term can be constructed, it is guaranteed to be well formed. And if
  it is a goal of a proved theorem (or lemma) it is definitely valid based on
  soundness of rules on that formal systems.
\item The repository of phometa is always in a consistent state. Phometa is
  quite cautious when the repository is being modified. For example, theorem can
  apply only a rule or a lemma that has been locked i.e.\ it is impossible that
  its dependencies will be changed. Another example is that when a node is
  deleted, Phometa will delete all of the nodes that depend on it as
  well\footnote{Of course, it will ask for confirmation first whether user want
    delete all of these or not.}. This is opposite to text-based proof
  assistants where user has full control over the repository, and if the
  repository is an inconsistent, the compiler will raise an error and user would
  fix it.
\item Lemmas allow reuse of proofs so there is no need for duplication. Users
  can select to do forward style proving (lots of small lemmas as steps of a
  proof) or backward style proving (a few big theorems).
\item It supports unicode input method and doesn't have reserved words so formal
  systems can be constructed in a more mathematical friendly environment.
\item It is a web-application so it can run on any machine that supports a
  web-browser. One might argue that it required Python for the back-end but most
  machines support Python out of the box anyway.
\end{itemize}

\vspace{1em}

\section{Limitation}

\newcommand{\hrefFutureWork}{\hyperref[sec:future_work]{future work}}

\begin{itemize}
\item It is hard to extend formal systems at the moment because Phometa doesn't
  allow grammar to inherit choices from another grammar. If a user wanted to
  extend a formal system, they would need to create a new one from scratch. For
  example, first order logic cannot be built from the existing propositional
  logic. This can be solved by making grammars extensible as described in
  \hrefFutureWork.
\item Phometa doesn't support automation well i.e.\ when a user constructs a
  proof, they need to say which rule or which lemma should be used. Guessing
  each step and automating the tree is possible. Mainstream proof assistants
  such as Coq and Isabelle have done it, however, it requires lots of heuristics
  and cleaver tricks, this is unrealistic to implement due to project time frame
  but it is a good consideration for \hrefFutureWork.
\item Each web-browser supports different sets of keyboard shortcuts. It is very
  hard for Phometa to find such keystrokes that are not visible characters and
  not keyboard shortcuts of any web-browser. So I ended-up using \pkbd{Alt}
  combined with a visible character to create a Phometa shortcut. This might
  have unwanted side effects but at least it work reasonably well with Google
  Chrome\footnote{To be precise, Chromium web-browser.} under a condition that
  the window containing Phometa has only one tab, so it will not suffer from
  \pkbd{Alt-\{1..9\}} which are used for switching tabs. However, this is not
  such a serious problem since user can always click a command in the keymap
  pane directly. Alternatively, this can be solved completely by making the
  keyboard shortcut customisable e.g.\ if some keyboard shortcut crash and user
  really want to use it, they can bind this command to another keystroke. Again,
  it is a good consideration for \hrefFutureWork.
\item The entire repository must be loaded into Phometa when it starts. This
  impacts scalability where repository is large since JavaScript can run out of
  memory. This is not usually a problem of text-based proof assistants since it
  verify a theorems one by one and doesn't need to put everything in memory at
  once.
\item Phometa requires a user to start a local server for individual use. It
  doesn't have a proper server where user can enter a link to and use it
  directly. To implement such a proper server, it requires user accounts and a
  database to manage users repositories, although it is possible to implement
  but it seems to overkill method and doesn't match project objective. It had
  lower priority than other features and can be considered as \hrefFutureWork.
\item Directly modifying \texttt{repository.json} before it is loaded into
  Phometa could result in undefined behaviour. This is because Phometa currently
  doesn't have any mechanism to verify consistency of repository before it will
  be loaded. This shouldn't cause any problem if user only interacts with
  repository via Phometa interface and does not try to hack the repository file
  directly. To solved this problem, we can write a function to verify
  consistency of loading repository as discussed further in \hrefFutureWork.
\end{itemize}

\section{Lesson Learnt}

Time management for a research project is one of the many things that I have
learnt during this project. I learnt that tasks always take time twice or thrice
longer than expected so it is vital to leave plenty of time before the deadline.
More importantly, I learnt that a better idea of feature always comes after we
start to implement something. It is quite hard to decide whether Phometa should
include some certain feature or not. It is about a trade-off between usefulness
of the feature and the risk of the project not being finished in time. These
kind of features usually came near the end of implementation where I knew
exactly what Phometa should be. This is bad because if I accepted the feature,
this would take sometime to implement and to edit the related part of this
report, which in turn, would impact the entire schedule of the plan. So I
usually took it as future work as described in next section.

I also learnt to believe in myself being capable to build something I dream of.
Formal proof always was my favourite topic since I studied Logic in the first
year. One day, I was drawing a derivation tree for a coursework, I had the idea
of this project. At the first time it seemed too scary because it is about
building a proof assistant from scratch, however after evaluated proof of
concept, it turned out to be feasible. So I decided to start it and approached
my supervisor.

Most importantly, I learnt many thing regarding to formal proof from this
project which is relevant to the topic that I want to do for PhD (Dependent Type
Theory). This gave me more familiarity and confidence in that field. Oppositely,
the curiosity on the field motivated me to work on this project better since I
know that this kind of knowledge gained during the project will be useful later
for sure.

\section{Future Works}
\label{sec:future_work}

Although Phometa has been designed and implemented up to a satisfactory level,
there is still plenty of room for improvements. For example,
\begin{itemize}
\item \textbf{Making Grammars extensible} --- Grammars should be able to extend another
  grammar in similar manner to how class in Object Oriented Programming extend
  other classes. For example, grammar of proposition in First Order Logic could
  extend grammar of proposition in Propositional Logic. This allows polymorphism
  where a term of extending grammar could be proven using rules or lemmas
  related to the base grammar.
\item \textbf{Plain-text as alternative term input method} --- When constructing
  a term, user should be able to write some terms in plain-text and Phometa will
  try to parse it according to formats of that grammar.
\item \textbf{Making Theorems construction become more automatics} --- When
  constructing a theorem, Phometa should be able to guess what rules or lemmas
  could be applied next (similar to Isabelle).
\item \textbf{Importation between Modules} --- Modules should be able to import some
  nodes from other modules, this improving scalability since a formal system can
  be expanded across multiple modules.
\item \textbf{Exportation to \LaTeX} --- Grammars, Rules, Theorems, and Proofs
  should be exportable to \LaTeX\ source code. So it can be used further in
  other documents such as reports or presentations.
\item \textbf{Repository Verification on Loading} --- Creating a function that
  checks whether the repository is in a valid state or not. Normally, it is not
  necessary as Phometa will always be in a consistent state, but when you want
  other people to use a formal system by obtaining \texttt{repository.json}, it
  is not guaranteed that someone will not modify that repository directly. So it
  is better to have a mechanism to check consistency of the repository anyway.
\end{itemize}

There are also other minor improvements such as
\begin{itemize}
\item Adding new formal systems to the standard library.
\item Making undo buttons.
\item Making notification message become more informative.
\item Deploy phometa to a proper server.
\item Making user interface become more mobile friendly.
\item Making a theme and keyboard shortcut become configurable.
\end{itemize}

\section{Conclusion}

At the end of this project. Phometa has been designed and been implemented up to
the level that is ready to use by anybody with a decent standard library and
tutorial. This, in turn, satisfies all of the objectives stated in the
introduction chapter. In addition, I also believe that Phometa on this state can
be a potential replacement for derivation-tree's manual-drawing so people don't
have to suffer from it tedious process and error prone anymore.

\end{document}