\documentclass[master.tex]{subfiles}
\begin{document}

\chapter{The Phometa Repository}
\label{chap:code_repository}

\section{Installation and Startup}
\label{sec:installation_startup}

You can download the compiled version of Phometa at

{\centering\url{https://github.com/gunpinyo/phometa/raw/master/build/phometa.tar.gz}}

Once you unzip this file, you see several files as the following

\begin{itemize}
\item \texttt{\textbf{phometa.html}} --- the main file containing Phometa
  web-based application.
\item \texttt{\textbf{style.css \& style.css.map}} --- control the layout and
  theme of \texttt{\textbf{phometa.html}}
\item \texttt{\textbf{naive.js}} --- JavaScript code that will be injected to
  \texttt{\textbf{phometa.html}}
\item \texttt{\textbf{repository.json}} --- contain proofs repository, can be
  loaded or saved by \texttt{\textbf{phometa.html}}
\item \texttt{\textbf{logo.png}} --- image that will be used as favicon in
  \texttt{\textbf{phometa.html}}
\item \texttt{\textbf{phometa-server.py}} --- python script that can be executed
  as local server.
\item \texttt{\textbf{phometa-doc.pdf}} --- this report, included here as the
  tutorial and reference.
\end{itemize}

You can start Phometa server by execute

\texttt{./phometa-server.py 8080}

where \texttt{8080} is port number, you can change this to another port number
if you like. Please note that Python is required for this server.

Then open your favourite web-browser\footnote{but Google Chrome is recommended}
and enter

\texttt{http://localhost:8080/phometa.html}

The program will look like figure \ref{fig:specification-phometa-home-window}

\section{The Anatomy of Phometa Repository}

Phometa is been developed using \emph{Git} as the version control and hosted
at

\url{https://github.com/gunpinyo/phometa}.

The Phometa repository \footnote{Please do not confuse with proofs repository
  which is the root package of proofs content} has the structure as the
following

\begin{itemize}
\item \texttt{\textbf{build/}} --- contains compiled version of Phometa and PDF
  version of this report.

\item \texttt{\textbf{doc/}} --- contains source code and supplementary images
  of this report.

\item \texttt{\textbf{scripts/}} --- contains scripts that can compile source
  code, run test suites, etc.

\item \texttt{\textbf{src/}} --- contains the entire source code of Phometa
  \begin{itemize}
  \renewcommand{\labelitemii}{$\star$}
  \item \texttt{\textbf{Naive/}} --- contains naive JavaScript functions that
    Elm functions can call.
  \item \texttt{\textbf{Tools/}} --- contains additional functions that are
    useful in general.
  \item \texttt{\textbf{Models/}} --- contains dependencies of \elmtt{Model} and
    its utilities.
  \item \texttt{\textbf{Updates/}} --- contains dependencies of \elmtt{updates}
    and its utilities.
  \item \texttt{\textbf{Views/}} --- contains dependencies of \elmtt{view}
    and its utilities.
  \item \texttt{\textbf{Main.elm}} --- the main entry of Phometa that
    wire up MCV components.
  \item \texttt{\textbf{repository.json}} --- initial proofs repository that
    contains standard library.
  \item \texttt{\textbf{phometa-server.py}} --- a python scripts back-end server.
  \item \texttt{\textbf{naive.js}} --- JavaScript that will be ship with
    compiled version of Phometa.
  \item \texttt{\textbf{style.scss}} --- layout and theme of Phometa.
  \end{itemize}
\item \texttt{\textbf{tests/}} --- contains tests suites
\item \texttt{\textbf{.gitignore}} --- specification of files that will be
  ignored by Git.
\item \texttt{\textbf{.travis.yml}} --- meta data regarding to Travis continuous
  integration.
\item \texttt{\textbf{LICENSE}} --- 3-clause BSD license.
\item \texttt{\textbf{README.md}} --- summary information for of this repository.
\item \texttt{\textbf{elm-package.json}} --- meta data regarding to Elm
  dependencies.
\item \texttt{\textbf{logo.png}} --- logo that will be used for favicon.

\end{itemize}

Please note that some files are omitted here due to limited space.

\newpage


\section{Building a Compiled Version}
We can build the compiled version of Phometa by go to the top directory of
Phometa repository then execute \texttt{./scripts/build.sh}. This will process
as the following
\begin{itemize}
\item create \texttt{build/} (if doesn't exist) or clean it (if doesn't empty)
\item compile(\LaTeX) the document in \texttt{doc/} and copy the main PDF file to \texttt{build/}
\item compile(Sass) \texttt{src/style.scss} to \texttt{build/style.css}
\item copy the following files to \texttt{build/}
  \begin{itemize}
  \item \texttt{src/naive.js}
  \item \texttt{src/repository.json}
  \item \texttt{src/phometa-server.py}
  \item \texttt{logo.png}
  \end{itemize}
\item compile(Elm) \texttt{src/Main.elm} to \texttt{build/phometa.html}
\item compress everything in \texttt{build/} to \texttt{build/phometa.tar.gz}
\end{itemize}

Please note that Elm compilation will consult \texttt{elm-package.json} for meta
data. The dependency of target Elm file will be compiled recursively. It can be
outputted as HTML file (for standalone application) or JavaScript file (to
embedded it as an element of bigger application).

\section{Back-End Communication}
There is a python server template from
\url{https://gist.github.com/UniIsland/3346170} which allows client to
download and upload local files directly in the server.

This server template, with some modification, allows us to create a local server
to serve \texttt{phometa.html} which is the main application that user will
interact with. \texttt{phometa.html}, in turn, will load further resource such
as \texttt{naive.js} (for naive code that can't be done directly in Elm) and
\texttt{style.css} (for theme of Phometa).

Once the setup is completed, Phometa will automatically load the repository by
sending Ajax request to load \texttt{repository.json} from the server, and then,
it will decode this json file to get instance of \elmtt{Package} and will set
this to \elmtt{model.root_package}.

Most of operation after this doesn't require further communication with the
server. Nevertheless, if user would like to save the repository, Phometa will
encode \elmtt{model.root_package} and send Ajax request to upload this as
\texttt{repository.json}.

\end{document}
