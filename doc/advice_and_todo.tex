\documentclass[master.tex]{subfiles}
\begin{document}

\section{advice and todo (need to remove this before final report due)}

\begin{itemize}
\item check that the last item point of achievement of introduction match with
  the correct tutorial chapters.
\item correct grammar in chapters 2,3,4,5,7,8
\item minor correction on chapters 4 and 7
\item finish lambda calculus on chapters 5 and apppendix
\item merge chapter 8 and 9 to become ``Evalulation and Conclusion''
\end{itemize}


\subsection{advice from Dr Krysia Broda}
for interim report
\begin{itemize}
  \item background can be done later, don't worry.
  \item for introduction, there should be an example to give an initial idea
    what is program will look like.
  \item for examples, really need to give more information for what current
    piece of example do, this is just for when the new idea has been introduce
    for each example. Examples will be on several chapters to give intuition
    along main story.
  \item specification, this should spilt into several chapters (perhaps lift
    specification from chapter to part)
  \item specification, again, need to explain every briefly topdown (package ->
    module -> node -> attribute -> placeholder),  then we can explain more
    detail with bottom up
  \item placeholder section on specification, it should be rewritten as a
    paragraph not bullet point since this specific case, it is easier to read
    as paragraph
  \item need to write high-level/low-level implementation chapters, this can be
    done along with real implementation which force myself to have a clear
    organisation every time.
  \item project plan --- preliminaries, need explain more what I have done so
    far
  \item we should focus on implementation first, specification could change
    later and we would have more time for if something is wrong with
    implementation
  \item screenshot in general shouldn't have dark theme as when they print out
    it will be unreadable.
\end{itemize}

\end{document}