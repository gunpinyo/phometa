\documentclass[master.tex]{subfiles}
\begin{document}
\chapter{Implementation}

This part we will talk about the implementation of phometa which is written in elm
hosted at \url{https://github.com/gunpinyo/phometa}.

\section{Decision on programming language}
Elm\supercite{elm-official-website} is a functional reactive programming
language. It allows programmer to create web application by declaratively
coding in Haskell-like language then compile the program to JavaScript. For more
information, please see elm official website at \url{elm-lang.org}.

One of the most attractive feture of elm is its reactivity. This idea introduces
a new data type called ``signal'' which is a data type that can change over
time. For example, let $c$ be a signal of integer defined as $a + b$ where $a$
and $b$ are other signal of  integers. If $a = 2$ and $b = 3$, then surely
$c = 5$. If later $a$ is change to $4$, then $c$ will got automatically updated
to $7$.

Reactivity work very well with functional paradigm since all variables are
immutable, so it is impossible for the program to be in inconsistent state in
the sense that programmer forget to update value. In fact, this can lead to a
good fit of model-controller-view (MCV) architecture.

Here are summery of reasons why I choose elm to implement phometa,
\begin{itemize}
  \item Phometa is a web application, and elm is created to build something like
    this
  \item Phometa mainly dealing with declarative object, it is better to use
    funcation  language to build it.
  \item Elm by its very nature, leads to MCV architecture which is good for
    application like phometa.
\end{itemize}

\section{Model-Controller-View Architecture}
  also talk about phometa modules hierarchy as well
\section{Modes and Keymap}
  how modes work and interact with keymap
\section{Examples of code --- Pattern Matching}
provide a full explanation of this part of code it is very interesting and small
enough to show some aspect of functional programming
\section{Compilation to Javascript, Html, and Css}
\section{Testing / Continious Integration}

% at run-time, the only thing that can change is signal which together
% can form dependency graph , programmer
% simply link each signal


% It allows programmer to write
% ac code in functional style similar to Haskell

% This fit with

% \chapter{Architecture}



\end{document}