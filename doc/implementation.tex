\documentclass[master.tex]{subfiles}
\begin{document}
\chapter{Implementation}
\label{chap:implementation}

This chapter aims to illustrate the high-level implementation of Phometa. So
that the reader will get a rough idea on how Phometa works.

\section{Decision on programming language}
Elm\supercite{elm-official-website} is a functional reactive programming
language. It allows programmers to create web applications by declaratively
coding in Haskell-like language then compiling the program to JavaScript.

One of the most attractive features of Elm is its reactivity. This idea
introduces a new data type called ``Signal''\footnote{Signal has been removed
  from Elm version 0.17 recently, however, Phometa uses Elm version 0.16 so it
  is fine to talk about Signal.} which is a data type that can change over time.
For example, \elmtt{Mouse.isDown : Signal Bool} represents a Boolean that holds
``True'' when the mouse is pressed, \elmtt{Window.dimensions : Signal (Int,
  Int)} represents a pair of integers that hold current width and height of the
window respectively. Custom Signal can be created using higher order functions
as in this example,
\begin{lstlisting}[language=elm]
Signal.map  : (a -> b) -> Signal a -> Signal b

window_area : Signal Int
window_area = Signal.map (\(w, h) -> w * h) Window.dimensions
\end{lstlisting}
\elmtt{window_area} is a signal that represents the area of current window. If
you resize the window, this value will change on real time. This is because when
the dimensions of the window is changed, it will notify
\elmtt{Window.dimensions} to update its value, \elmtt{Window.dimensions} in
turn, notifies \elmtt{window_area} to update its value as well.

These Signals can link together to form a dependency graph, eventually we will
create \elmtt{main : Signal Html} which is a HTML representation that can change
over time, Elm will detect this \elmtt{main} and render it in a web-browser. For
example,

\begin{lstlisting}[language=elm]
import Mouse
import Signal exposing (Signal)
import Html exposing (Html, div, text)
import Html.Attributes exposing (style)

view : Bool -> Html
view is_clicked =
  if is_clicked then div [style [("color", "green")]]
                         [text "Clicked !!"]
                else text "Please click this page."

main : Signal Html
main = Signal.map view Mouse.isDown
\end{lstlisting}

Once this code is compiled, we will get a HTML file (with JavaScript embedded)
such that if it is opened in a web-browser, it will show text ``Please click
this page.''. If you click somewhere in the page, the text will change to
``Clicked !!'' with green background colour, which will change back when you
release the mouse.

The example above shows that Elm could replace all of HTML, JavaScript, and even
CSS. In other words, the entire front-end development could be done in one
purely functional language. Therefore, Elm is a good candidate to be used as the
main programming language for web-based application like Phometa.

For more information, please see Elm official website at \url{elm-lang.org}.

\section{Model-Controller-View Architecture}

Recall from the previous example, you could see that the HTML will depend
directly on corresponding Signal(s) but in real applications we need a way to
store variables, which is achievable by the following function

\begin{lstlisting}[language=elm]
Signal.foldp : (a -> b -> b) -> b -> Signal a -> Signal b
\end{lstlisting}

Initially, the Signal generated form \elmtt{Signal.foldp} will have a value the
same as the second argument, then whenever the Signal from the third argument is
updated (possibly with the same value), it will be \emph{reduced} with the
current output Signal in order to get new output Signal. For example,

\begin{lstlisting}[language=elm]
click_total : Signal Int
click_total =
  let fold_func clicked acc = if clicked then acc + 1 else acc
   in Signal.foldp fold_func 0 Mouse.isDown
\end{lstlisting}

\elmtt{click_total} is a Signal that store an integers. Initially, it starts
with $0$. When user clicks mouse on the page, \elmtt{Mouse.isDown} will notify
\elmtt{click_total} to invoke \elmtt{fold_func}, and the value of this Signal
will be updated to $1$. If you click it again it will be $2$, $3$, $4$, etc.

In Phometa, an accumulator of \elmtt{Signal.foldp} is \elmtt{Model} represents
the global state of the entire program. The third argument (input Signal) of
\elmtt{Signal.foldp} is \elmtt{Action} representing the Signal that changes
every time when user presses the keyboard or clicks a button with Phometa. We
could define the main entry of Phometa program as the following\footnote{Phometa
  code represented here is just a simplify version, I can't show the real one
  because MCV architecture is modified to support back-end communication so it
  is harder to understand.}

\begin{figure}[H]
\begin{framed}
\begin{lstlisting}[language=elm]
module Main where

import Html exposing (Html)
import Keyboard exposing (keysDown)
import Set exposing (Set)
import Models.Model exposing (Model)
import Models.ModelUtils exposing (init_model)
import Models.Action exposing (Action (..) , mailbox, address)
import Updates.Update exposing (update)
import Views.View exposing (view)

keyboard_signal : Signal Action
keyboard_signal = Signal.map
  (Set.toList >> List.sort >> ActionKeystroke) keysDown

action_signal : Signal Action
action_signal = Signal.merge mailbox.signal keyboard_signal

model_signal : Signal Model
model_signal = Signal.foldp update init_model action_signal

main : Signal Html
main = Signal.map view model_signal
\end{lstlisting}
\end{framed}
\caption{Simplified version of \texttt{Main.elm}}
\label{fig:implementation-simplified-main}
\end{figure}

By using \elmtt{Signal.foldp} together with \elmtt{update : Action -> Model ->
  Model}, we achieve \elmtt{model_signal} which is a Signal represented the
current global state of Phometa, which, in turn, can be transformed using
\elmtt{Signal.foldp} together with \elmtt{view : Model -> Html} to achieve
\elmtt{main} which represents the current HTML page.

Now, it becomes clear that Model-Controller-View architecture could be separated
by \elmtt{Model}, \elmtt{update : Action -> Model -> Model}, and \elmtt{view :
  Model -> Html} respectively.

\newpage

\section{Model, Command, Keymap, and Action}
\elmtt{Model} is defined in \texttt{src/Models/Model.elm} as the following

\begin{figure}[H]
\begin{framed}
\begin{lstlisting}[language=elm]
type alias Model =
  { config       : Config
  , root_package : Package
  , root_keymap  : Keymap
  , grids        : Grids
  , pane_cursor  : PaneCursor
  , mode         : Mode
  , message_list : MessageList
  , environment  : Environment
  }
\end{lstlisting}
\end{framed}
\caption{Declaration of \elmtt{Model}}
\label{fig:implementation-model}
\end{figure}

A function that can manipulate \elmtt{Model} has type signature as
\begin{lstlisting}[language=elm]
type alias Command = Model -> Model
\end{lstlisting}

\elmtt{Action} is defined in \texttt{src/Models/Action.elm}
as the following

\begin{figure}[H]
\begin{framed}
\begin{lstlisting}[language=elm]
type Action
  = ActionNothing
  | ActionCommand Command
  | ActionKeystroke Keystroke
\end{lstlisting}
\end{framed}
\caption{Declaration of \elmtt{Action}}
\label{fig:implementation-action}
\end{figure}

One way to interact with Phometa is to click some button (or any clickable
area). When this is clicked, it will update the value of \elmtt{action_signal}
to ``\elmtt{ActionCommand cmd}'' where \elmtt{cmd} is an instance of
\elmtt{Command} injected to that button earlier.

\elmtt{Signal.foldp} will the detect change in \elmtt{action_signal} so it
applies function \elmtt{update} with ``\elmtt{ActionCommand cmd}'' to
\elmtt{model_signal}. The function \elmtt{update}, in turn, will apply command
\elmtt{cmd} to current model and this is how the model can be updated.

Again, \elmtt{Signal.map} will the detect change in \elmtt{action_model} so it
applies function \elmtt{view} with current model to get latest HTML instance.
Please note that \elmtt{view} might inject further commands to buttons; if user
clicks any button, then the entire process will happen again.

Another way to interact with Phometa is to press some keystroke, this will make
\elmtt{action_signal} changes to ``\elmtt{ActionKeystroke keystroke}'', where
\elmtt{keystroke} is a \elmtt{List} of \elmtt{KeyCode} obtained from
\elmtt{Keyboard.keysDown}.

\elmtt{Signal.foldp} will detect the change in \elmtt{action_signal} so it
applies function \elmtt{update} with ``\elmtt{ActionKeystroke keystroke}'' to
\elmtt{model_signal}. The function \elmtt{update}, in turn, will search a
command that corresponds to this keystroke from \elmtt{model.root_keymap} to
apply such a command with the current model. Now, the remaining process is the
same as when we click button.

\section{Structure of Proofs Repository}
In this section, we will show the complete internal representation of a proofs
repository. All codes inside this section come from
\texttt{src/Models/RepoModel.elm}

Let's define some common data-types\footnote{\elmtt{Arguments} depends on
  \elmtt{RootTerm} which will be defined later in this section} which will be
used by later data-types.

\begin{figure}[H]
\begin{framed}
\begin{lstlisting}[language=elm]
type alias ContainerName = String
type alias ContainerPath = List ContainerName
type alias Format = String
type alias VarName = String
type alias Parameters = List { grammar  : GrammarName
                             , var_name : VarName }
type alias Arguments = List RootTerm
\end{lstlisting}
\end{framed}
\caption{Representation of common data-types that will be used later on}
\label{fig:implementation-repo-common}
\end{figure}

Now, we can define a proof repository which is just a big \elmtt{Package} stored
in the global model named \elmtt{model.root\_package} and can be defined as the
following

\begin{figure}[H]
\begin{framed}
\begin{lstlisting}[language=elm]
type alias PackageName = ContainerName
type alias PackagePath = List PackageName
type alias Package = { is_folded : Bool
                     , dict : Dict ContainerName PackageElem }
type PackageElem
  = PackageElemPkg Package
  | PackageElemMod Module
\end{lstlisting}
\end{framed}
\caption{Representation of \elmtt{Package} and related components.}
\label{fig:implementation-repo-package}
\end{figure}

We could think of a \elmtt{Package} roughly as a collection containing other
\elmtt{Package}s and \elmtt{Module}s. We also need \elmtt{PackagePath} to
reference a package relative to root package.

\elmtt{Module} is a collection containing nodes. \elmtt{ModulePath} is a
unique reference from root package to this module. Please note that
\elmtt{Module} uses \elmtt{OrderedDict} in contrast to \elmtt{Package} which
uses \elmtt{OrderedDict}; this is because \elmtt{Module} needs to remember order
of nodes but \elmtt{Package} simply shows elements alphabetically.

\begin{figure}[H]
\begin{framed}
\begin{lstlisting}[language=elm]
type alias ModuleName = ContainerName
type alias ModulePath = { package_path : PackagePath
                        , module_name  : ModuleName }
type alias Module = { is_folded : Bool
                    , nodes     : OrderedDict NodeName Node }
\end{lstlisting}
\end{framed}
\caption{Representation of \elmtt{Module} and related components.}
\label{fig:implementation-repo-module}
\end{figure}

There are four kinds of \elmtt{Node} which are \elmtt{Comment}, \elmtt{Grammar},
\elmtt{Rule}, and \elmtt{Theorem}\footnote{\elmtt{Bool} that follows
  \elmtt{NodeTheorem} tells whether that theorem has been converted to lemma or
  not.}; each of them will have corresponding data-type which will be defined
later. \elmtt{NodePath} is a unique reference from root package to this node.

\begin{figure}[H]
\begin{framed}
\begin{lstlisting}[language=elm]
type alias NodeName = ContainerName

type alias NodePath =
  { module_path : ModulePath
  , node_name   : NodeName
  }

type Node
  = NodeComment String
  | NodeGrammar Grammar
  | NodeRule Rule
  | NodeTheorem Theorem Bool

type NodeType
  = NodeTypeComment
  | NodeTypeGrammar
  | NodeTypeRule
  | NodeTypeTheorem
\end{lstlisting}
% type alias NodeName = ContainerName
% type alias NodePath = { module_path : ModulePath
%                       , node_name   : NodeName }
% type Node = NodeComment String
%           | NodeGrammar Grammar
%           | NodeRule Rule
%           | NodeTheorem Theorem Bool
% type NodeType = NodeTypeComment
%               | NodeTypeGrammar
%               | NodeTypeRule
%               | NodeTypeTheorem
% \end{lstlisting}
\end{framed}
\caption{Representation of \elmtt{Node} and related components.}
\label{fig:implementation-repo-node}
\end{figure}

\newpage

We could define \elmtt{Grammar} as the following figure which is coincided to
\emph{Node Grammar} described in section \ref{spec:node_grammar}.

\begin{figure}[H]
\begin{framed}
\begin{lstlisting}[language=elm]
type alias GrammarName = String

type alias Grammar =
  { is_folded     : Bool
  , has_locked    : Bool
  , metavar_regex : Maybe Regex
  , literal_regex : Maybe Regex
  , choices       : List GrammarChoice
  }

type alias GrammarChoice = StripedList Format GrammarName
\end{lstlisting}
% type alias GrammarName = String
% type alias Grammar = { is_folded     : Bool
%                      , has_locked    : Bool
%                      , metavar_regex : Maybe Regex
%                      , literal_regex : Maybe Regex
%                      , choices       : List GrammarChoice }
% type alias GrammarChoice = StripedList Format GrammarName
% \end{lstlisting}
\end{framed}
\caption{Representation of \elmtt{Grammar} and related components.}
\label{fig:implementation-repo-grammar}
\end{figure}

\vspace{-1em}

\elmtt{StripedList Format GrammarName} is just a pair of ``list of
\elmtt{Format}'' and ``list of \elmtt{GrammarName}'' but it also ensure that
length of the former is longer than the latter by exactly 1.

Next is a term, which can be defined like this,

\begin{figure}[H]
\begin{framed}
\begin{lstlisting}[language=elm]
type Term
  = TermTodo
  | TermVar VarName
  | TermInd GrammarChoice (List Term)

type alias RootTerm =
  { grammar : GrammarName
  , term : Term
  }

type VarType
  = VarTypeMetaVar
  | VarTypeLiteral
\end{lstlisting}
% type Term = TermTodo
%           | TermVar VarName
%           | TermInd GrammarChoice (List Term)
% type alias RootTerm = { grammar : GrammarName
%                       , term : Term }
% type VarType = VarTypeMetaVar
%              | VarTypeLiteral
% \end{lstlisting}
\end{framed}
\caption{Representation of \elmtt{Term} and related components.}
\label{fig:implementation-repo-term}
\end{figure}

We usually use \elmtt{RootTerm} as a term at the top level, this is because it
know its own grammar. In fact, \elmtt{Term} is just an auxiliary model of
\elmtt{RootTerm}. Please note that it is possible to find the grammar of every
sub-term of a \elmtt{RootTerm}, because we can refer to \elmtt{grammar}
property of \elmtt{RootTerm} for top level and refer to \elmtt{GrammarChoice}
of \elmtt{TermInd} for deeper sub-term.

Once \elmtt{RootTerm} has been defined, we can define \elmtt{Rule} and
\elmtt{Theorem} as the following figures which are coincided to \emph{Node Rule}
and \emph{Node Theorem} described in sections \ref{spec:node_rule} and
\ref{spec:node_theorem}, respectively.

\begin{figure}[H]
\begin{framed}
\begin{lstlisting}[language=elm]
type alias RuleName = String

type alias Rule =
  { is_folded       : Bool
  , has_locked      : Bool
  , allow_reduction : Bool
  , parameters      : Parameters
  , conclusion      : RootTerm
  , premises        : List Premise
  }

type Premise
  = PremiseDirect RootTerm
  | PremiseCascade (List PremiseCascadeRecord)

type alias PremiseCascadeRecord =
  { rule_name : RuleName
  , pattern : RootTerm
  , arguments : Arguments
  , allow_unification : Bool
  }
\end{lstlisting}

% type alias RuleName = String
% type alias Rule = { is_folded       : Bool
%                   , has_locked      : Bool
%                   , allow_reduction : Bool
%                   , parameters      : Parameters
%                   , conclusion      : RootTerm
%                   , premises        : List Premise }
% type Premise = PremiseDirect RootTerm
%              | PremiseCascade (List PremiseCascadeRecord)
% type alias PremiseCascadeRecord = { rule_name : RuleName
%                                   , pattern : RootTerm
%                                   , arguments : Arguments
%                                   , allow_unification : Bool }
% \end{lstlisting}
\end{framed}
\caption{Representation of \elmtt{Rule} and related components.}
\label{fig:implementation-repo-rule}
\end{figure}

\vspace{-1.5em}

\begin{figure}[H]
\begin{framed}
\begin{lstlisting}[language=elm]
type alias TheoremName = String

type alias Theorem =
  { is_folded : Bool
  , goal      : RootTerm
  , proof     : Proof
  }

type Proof
  = ProofTodo
  | ProofTodoWithRule RuleName Arguments
  | ProofByRule RuleName Arguments PatternMatchingInfo
      (List Theorem)
  | ProofByLemma TheoremName PatternMatchingInfo
\end{lstlisting}
\end{framed}
\caption{Representation of \elmtt{Theorem} and related components.}
\label{fig:implementation-repo-theorem}
\end{figure}

\elmtt{PatternMatchingInfo} is a result when a pattern matching is successful
and can be defined as the following,


\begin{figure}[H]
\begin{framed}
\begin{lstlisting}[language=elm]
type alias SubstitutionList =
  List { old_var : VarName
       , new_root_term : RootTerm
       }

type alias PatternMatchingInfo =
  { pattern_variables : Dict VarName RootTerm
  , substitution_list : SubstitutionList
  }
\end{lstlisting}
\end{framed}
\caption{Representation of \elmtt{PatternMatchingInfo} and related components.}
\label{fig:implementation-repo-pattern-matching-info}
\end{figure}

As we know that terms that match to the same pattern variable will be unified
during matching, \elmtt{SubstitutionList} tell us that what meta-variables
(\elmtt{old\_var}) should be replaced \elmtt{new\_root\_term}. The order of
substitution is important as later substitution might depend on earlier one. Now
we can guarantee that, after unification, each pattern variable will match to
only one term, this make it possible to create a dictionary
\elmtt{pattern\_variables} that map these pattern variables to the corresponded
terms.

\section{Pattern Matching and Unification}
Although it is impossible to show all of codes written in Phometa, there is a
faction of code related to \emph{Pattern Matching} that is small enough to be
described entirely; it is also one of the most interesting part as well.

Please note that this section only described how pattern matching works. For the
supplementary source code, please see appendix chapter
\ref{chap:supplementary_source_code}.

Let start with the most important function, \elmtt{pattern\_match} is a function
that pattern match \elmtt{pattern} against \elmtt{target}, if success, it will
the return pattern matching information as described in previous section.
Another function is \elmtt{pattern\_match\_multiple} which is the same as
previous function but receive a list of pair of \elmtt{(pattern, target)} rather
than just one \elmtt{pattern} and one \elmtt{target}. These two are shown in
figure \ref{fig:supp-pattern-matching-info}.

Both of functions above call function \elmtt{pattern\_match\_get\_vars\_dict} that
perform the actual pattern matching and return a dictionary that map each
pattern variable to a list of terms. The the pattern match process is shown in
figure \ref{fig:supp-pattern-pattern-match-get-vars-dict} and can be
described as the following

\newpage

\begin{itemize}
\item If the grammar of pattern and target is not the same, fail immediately.
\item If any of pattern or target is unfinished term, fail immediately.
\item If pattern is a variable, succeed immediately with that variable
  mapped to the target.
\item If target is a variable but pattern is not, fail immediately (one-way
  matching only).
\item the only possibility left is both of them are
  \elmtt{TermInd}\footnote{A term can be either unfinished (\elmtt{TermTodo}),
    variable (\elmtt{TermVar}), or
    \elmtt{TermInd}}, so it can process as the following
  \begin{itemize}
  \item If the grammar-choice of pattern and target is not the same, fail
    immediately.
  \item Otherwise, construct $n$ pairs of (sub-pattern, sub-target) where
    \begin{itemize}
    \item $n$ is the number of sub-term of pattern which must the same as number of sub-term of target.
    \item $i^{th}$ sub-pattern is constructed from $i^{th}$ sub-term of pattern
      and $i^{th}$ sub-grammar in the grammar-choice.
    \item $i^{th}$ sub-target is constructed from $i^{th}$ sub-term of target
      and $i^{th}$ sub-grammar in the grammar-choice.
    \end{itemize}

    For each pair of (sub-pattern, sub-target), apply
    \elmtt{pattern\_match\_get\_vars\_dict} then it will return either
    pattern-to-list-of-terms dictionary or failure.

    \begin{itemize}
    \item If any of sub-result return as failure, fail immediately
    \item Otherwise, merge all dictionaries of sub-results together and
      succesfully return it.
    \end{itemize}

  \end{itemize}
\end{itemize}

After \elmtt{pattern\_match\_get\_vars\_dict} produces matching dictionary. \\
\elmtt{vars\_dict\_to\_pattern\_matching\_info} as shown in figure
\ref{fig:supp-vars-dict-to-pattern-matching-info} will try to fit this into
\elmtt{PatternMatchingInfo}. This can be done by finding a
\elmtt{SubstitutionList} that can eliminate all of ambiguity among terms which
are the value of matching dictionary. This \elmtt{SubstitutionList} can be found
by using the following method.
\begin{itemize}
\item For each pattern variable, check whether it is meta variable or literal.
\item If it is meta variable, unify\footnote{Unification process will be
    explained after this function.} corresponded terms to each other then append
  these unification result to the main \elmtt{SubstitutionList}.
\item If it is literal, check that each of term is that exact variable as well,
  otherwise, fail.
\end{itemize}

Then we can substitute this \elmtt{SubstitutionList} to the matching directly to
get \elmtt{pattern\_variables}, hence be able to return
\elmtt{PatternMatchingInfo} as expected.

Function \elmtt{unify} as described in figure \ref{fig:supp-unify} receives term
\elmtt{a} and \elmtt{b} then try to build the most general unifier (mgu) which
is the most general substitution list that is specific enough to make term
\elmtt{a} and \elmtt{b} identical. This function is similar to pattern matching
in \elmtt{pattern\_match\_get\_vars\_dict} but this is two way matching and it
is aware whether a variable is meta variable or literal.

Recall back to function \elmtt{vars\_dict\_to\_pattern\_matching\_info}, ``unify
the corresponded terms'' means we will find the most general substitution list
that can be used to substitute to each corresponded term and make every term
identical. This achievable by
\begin{itemize}
\item apply function \elmtt{unify} on the $1^{st}$ and $2^{nd}$ term, then we
  will get the first substitution list.
\item substitute the first substitution list to the $1^{st}$ and $3^{rd}$ terms,
  then apply function \elmtt{unify} to it, then append this substitution list to
  the first one to get the second substitution list.
\item substitute the second substitution list to the original $1^{st}$ and $4^{th}$
  terms, then apply function \elmtt{unify} to it, then append this substitution
  list to the second one to get the third substitution list.
\item keep repeat this pattern until the last term get substituted and unify, the
  latest substitution list will the most general substitution list that we want
  at the first place.
\end{itemize}

There are also other several auxiliary functions for pattern matching as
described in figure \ref{fig:supp-auxiliary} which can be described as the
following
\begin{itemize}
\item \elmtt{substitute} --- replace every occurrence of the specified variable
  ($1^{st} arg$) by the new term ($2^{st} arg$) inside the current term ($3^{rd}
  arg$).
\item \elmtt{multiple\_root\_substitute} --- substitute a root term by a
  substitution list.
\item \elmtt{pattern\_substitute} --- substitute a term by
  pattern-to-root-term dictionary.
\item \elmtt{pattern\_root\_substitute} --- the same as
  \elmtt{pattern\_substitute} but substitute to a root-term rather than a term.
\item \elmtt{pattern\_matching\_info\_substitute} --- update
  \elmtt{PatternMatchingInfo} by a substitution list.
\item \elmtt{pattern\_matchable} --- check whether the pattern root-term can be pattern
  matched against target root-term or not.
\item \elmtt{merge\_pattern\_variables\_list} --- merge all of
  pattern-to-list-of-terms dictionaries in to a single dictionary.
\end{itemize}

\section{Testing / Continious Integration}
Functional programming makes testing really easy. In order to test a function,
we need to create a test suite and call that function with several inputs then
check whether the outputs are the same as expected or not.

For example, we want to test a function \elmtt{list\_insert} as the following

\begin{figure}[H]
\begin{framed}
\begin{lstlisting}[language=elm]
list_insert : Int -> a -> List a -> List a
list_insert n x xs =
  if n <= 0 then x :: xs else
    case xs of
      []      -> [x]
      y :: ys -> y :: (list_insert (n - 1) x ys)
\end{lstlisting}
\end{framed}
\caption{Function \elmtt{list\_insert} from \texttt{src/Tools/Utils.elm}}
\label{fig:implementation-test-src}
\end{figure}

We could write a test suite for \elmtt{list\_insert} in the test suite of its module like this

\begin{figure}[H]
\begin{framed}
\begin{lstlisting}[language=elm]
module Tests.Tools.Utils where

import ElmTest exposing (Test, test, suite, assert,
                         assertEqual, assertNotEqual)
import Tools.Utils exposing (..)

tests : Test
tests = suite "Tools.Utils" [
  suite "list_skeleton" [...],
  suite "list_insert" [
    test "n < 0" <|
      assertEqual [7, 6, 4, 9] (list_insert (-1) 7 [6, 4, 9]),
    test "n = 0" <|
      assertEqual [7, 6, 4, 9] (list_insert 0 7 [6, 4, 9]),
    test "n in range" <|
      assertEqual [6, 4, 7, 9] (list_insert 2 7 [6, 4, 9]),
    test "n >= length" <|
      assertEqual [6, 4, 9, 7] (list_insert 10 7 [6, 4, 9]),
    ... ],
  suite "parity_pair_extract" [...],
  suite "remove_list_duplicate" [...],
  ... ]
\end{lstlisting}
\end{framed}
\caption{The test suite for module \texttt{src/Tools/Utils.elm}}
\label{fig:implementation-test-test}
\end{figure}

A test suite of each module will be included in the main test suite as the
following

\begin{figure}[H]
\begin{framed}
\begin{lstlisting}[language=elm]
module Main where

import Task
import Console exposing (run)
import ElmTest exposing (Test, suite, consoleRunner)

...
import Tests.Tools.Utils
...

tests : Test
tests = suite "Tests" [ ... , Tests.Tools.Utils.tests, ... ]

port runner : Signal (Task.Task x ())
port runner = run (consoleRunner tests)
\end{lstlisting}
\end{framed}
\caption{The main test suite which is at \texttt{tests/Tests/Main.elm}}
\label{fig:implementation-test-main}
\end{figure}

To run\footnote{The instruction of running test suite is shown here to give a
  favour to reader that how did I test my code, rather than asking user to run
  the test.} the main test suite, go to top level of Phometa repository\\ then
execute \texttt{./scripts/tests.sh} which will be processed as the following.
\begin{itemize}
\item compile \texttt{tests/Tests/Main.elm} to \texttt{raw-tests.js}
\item convert \texttt{raw-tests.js} to \texttt{tests.js} using external bash
  script\footnote{\url{https://raw.githubusercontent.com/laszlopandy/elm-console/master/elm-io.sh}}.
\item execute \texttt{tests.js} by \emph{Node.js}, this will show the testing result.
\end{itemize}

Please note that the test suite doesn't have 100\% coverage and functions
related to HTML are not testable using this method.

Last but not least, \emph{Travis CI} is used for continuous integration.
Basically, every time when something is pushed to \emph{GitHub}, Travis CI will
check \texttt{.travis.yml} this will result in the main test suite being
executed in virtual machine using similar testing process above.


\end{document}