
%%% Local Variables:
%%% mode: latex
%%% TeX-master: "master"
%%% End:

\chapter{Introduction}

In logic, there are lots of formal systems, each of them has pros and cons. However, most of them has some pattern in common. This project intends to show that these formal systems can be materialised using a common application which is called ``phometa''.

Phometa is a proof assistant that can build a formal system based on visualisation. It runs via web browser. This has advantage over traditional text-based proof assistant like Coq or Agda, such as

\begin{itemize}
    \item Users can use arbitrary string for syntax, hence, no more ``reserve words'' or ``name collusion''. It is possible to include Unicode or LaTeX for string as well.
    \item Terms must have a proper tree structure at the very beginning when user create it via input method, hence, no more ambiguity.
    \item Because we can control the way that user interact with phometa, so a formal system is alway in a valid state,
    \item Easier on eyes for new comers which encourage them to play more with phometa.
\end{itemize}

Phometa pronounces as ``po-met-ta'', which come from,

\begin{itemize}
    \item \textbf{pho}, it is the Thai name of a tree specie which has heart-shape. This reminded me about $\lambda$ shape.
    \item \textbf{meta}, because we want to build a system that describe other systems, word ``meta'' is suitable since it is related to meta-theory.
\end{itemize}