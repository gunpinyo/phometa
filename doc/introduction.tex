\documentclass[master.tex]{subfiles}
\begin{document}

\chapter{Introduction}

\section{Motivation}

Proofs are very important to all kinds of Mathematics because they ensure the
correctness of theorems. However, it is hard to verify the correctness of a
proof itself especially for a complex proof. To tackle this problem, we can
prove a theorem on a \emph{proof assistant}, aka \emph{interactive theorem
  prover}, which provides a rigorous method to construct a proof such that an
invalid proof will never occur. Therefore if we manage to complete a proof, it
is guaranteed that the proof is valid.

There are many powerful and famous proof assistants such as
Coq\supercite{coq-official-website}, Agda\supercite{agda-official-website}, and
Isabelle\supercite{isabelle-official-website} which are suitable for extreme use
case of complex proofs. Nevertheless, they have a steep learning curve and have
specific meta-theory behind it, for example, Coq has Calculus of Inductive
Construction (CIC), Agda has Unified Theory of Dependent
Types\supercite{norell:thesis}\supercite{Luo:1994:CRT:184757} which are quite
hard for newcomers. To solve this problem they should start with something
easier than these and come back again later.

One of the easiest starting point to learn about formal proof is to use
derivation trees where validity of a term is derived from a derivation rule
together with validity of zero or more terms depending on the rule. These
prerequisite terms can be proven similarly to the main term. The proving process
will happen recursively, this lead to tree-like structure of the final proof,
this is why it is called ``derivation tree''.

The naive way to construct such a derivation tree is to draw it on a paper,
however, this has many disadvantages such as
\begin{itemize}
\item The width of a derivation tree usually grow exponentially to its height ---
  hard to arrange the layout on a paper.
\item We don't know that how much space that each branch requires --- need to
  recreate the tree for many iterations.
\item Variables might need to be rewritten by other terms as a result of
  internal unification when applying a rule --- again, need to recreate the tree
  for many iterations.
\item When a derivation tree is completed, there is nothing to guarantee that it
  doesn't have any errors --- conflict with the ambition to use proof assistant
  at the first place.
\end{itemize}
So I decided to create a proof assistants called \emph{Phometa} to solve this
derivation-tree manually-drawing problem. To be precise, Phometa is proof
assistant that allows user to create a formal system and prove theorems using
derivation trees.

Phometa fundamentally consists of three kinds\footnote{There exists the fourth
  kind of node which are comment node but I don't include it there since it is
  not relevant to the fundamental concept.} of node as the following
\begin{itemize}
\item \emph{Grammar} (or Backus-Naur Form) --- How to construct a well-form
  term.\\ For example, a simple arithmetic expression can be constructed by a
  number \emph{or} two expressions adding together \emph{or} two expressions
  multiply together.
\item \emph{Rule} (or derivation rule) --- A reason that can be used to prove
  validity of terms. For example, $(u + v) = (x + y)$ is valid if $u = x$ and $v
  = y$.
\item \emph{Theorem} (or derivation tree) --- An evidence (proof) showing that \\
  a particular term is valid. For example,
  \begin{itemize}
  \item[] $((3 + 4) \times 5) = ((4 + 3) \times 5)$ is valid by rule
    \rulename{add-intro} and
    \begin{itemize}
    \item[] $(3 + 4) = (4 + 3)$ is valid by rule \rulename{add-comm}
    \item[] $5 = 5$ is valid by rule \rulename{eq-refl}
    \end{itemize}
  \end{itemize}
\end{itemize}
A formal system will be represented by a set of grammars and rules. Validity of
terms will be represented by theorems (derivation trees).

In term of usage, users can Phometa by either
\begin{itemize}
\item Learn one of many existing formal systems provided in Phometa's standard
  library and try to proof some theorem regarding to that formal system.
\item Create their own formal system or extend an existing formal system, then
  do some experiments about it.
\end{itemize}

In order to make Phometa easy to use, it is designed to be web-based
application. Users will interact with Phometa mainly by clicking buttons and
pressing keyboard-shortcut. This has advantages over traditional proof assistant
because it is easier to read, ill-from terms never occur, and guarantee that the
entire system is always in consistent state.

\section{Objectives}
\begin{itemize}
  \item To make a construction of derivation tree become more systematic. Hence, users become more productive and have less chance to make an error.
  \item To encourage users to create their own formal systems and reason about it.
  % \item To integrate operational semantics with in formal system. Hence, be able represent the entire logical system with in just one framework.
  \item To show that most of formal systems have a similar meta-structure which can be implemented using common framework.
  \item To show advantages of visualised proof assistant over traditional one.
\end{itemize}

\section{Achievement}
\begin{itemize}
\item Finished designing Phometa specification in such a way to keep it simple yet be able
  to produce a complex proof.
\item Finished implementing Phometa. All of basic functionality is working.
\item Encoded several formal systems such as Simple Arithmetic, Propositional Logic, and Typed Lambda Calculus
  as standard library in Phometa.
\item Wrote a tutorial for newcomers to use Phometa (chapters
  \ref{chap:background}, \ref{chap:example_simple_arithmetic},
  \ref{chap:example_propositional_logic}, and
  \ref{chap:example_lambda_calculus}).
\end{itemize}

\end{document}