\documentclass[master.tex]{subfiles}
\begin{document}

\chapter{Introduction}

\section{Motivation}

Proofs are very important to all kinds of Mathematics because they ensure the
correctness of theorems. However, it is hard to verify the correctness of a
proof itself especially for a complex proof. To tackle this problem, we can
proof a theorem on \emph{proof assistant}, aka \emph{interactive theorem
  prover}, which does not allow us to make a mistake for each step of the proof,
hence, when we finished writing a proof, we can make sure that it is correct.

There are many powerful and famous proof assistants such as
Coq\supercite{coq-official-website}, Agda\supercite{agda-official-website}, and
Isabelle\supercite{isabelle-official-website} which are suitable for extreme use
case of complex proofs. Nevertheless, they have steep learning curve and have
specific meta-theory behind it, for example, Coq has Calculus of Inductive
Construction (CIC), Agda has Unified Theory of Dependent
Types\supercite{norell:thesis}\supercite{Luo:1994:CRT:184757} which are quite
hard for newcomers. So I decided to create another proof assistants called
\emph{phometa} that is easy to use and require minimal prior knowledge.

To be precise, phometa is a tool to build a derivation tree. In order to achieve
this, phometa has 3 kinds of nodes
\begin{itemize}
\item \emph{Grammar} --- How to construct a well-form term. For example,
  proposition can be either
  $\top \quad \bot \quad A \wedge B \quad A \vee B \quad \neg A \quad A
  \rightarrow B \quad A \leftrightarrow B$ where $A$ and $B$ are other arbitrary
  propositions.
\item \emph{Rule} (or derivation rule) --- Reason that can be used to prove a
  term. For example, $A \wedge B$ is valid if we can prove that $A$ is valid and
  $B$ is valid.
\item \emph{Theorem} (or derivation tree) --- An evidence (proof) that show that
  a term is valid.
\end{itemize}
By this 3 kinds of nodes. We can build any formal system (consists of grammars
and rules) and prove its correctness as long as it support derivation tree.

\newpage

Users can phometa by either
\begin{itemize}
\item Learn one of many existing formal systems provided in phometa's standard
  library and try to proof some theorem regarding to that formal system.
\item Create their own formal system or extend an existing formal system, and do
  some experiments about it.
\end{itemize}
Once they get used to with this derivation tree builder. They can switch to more
powerful proof assistants as mention above.

In order to make phometa easy to use, it is designed to be web-based
application. Users will interact with phometa mainly by clicking buttons and
pressing keyboard-shortcut. This has advantages over traditional proof assistant
because it is easier to read, ill-from terms never occur, and guarantee that the
entire system is always in consistent state.

\section{Objectives}
This project carries several objective as the following
\begin{itemize}
  \item To make a construction of derivation-tree become more systematic. Hence, users become more productive and have less chance to make an error.
  \item To encourage users (especially students and newcomers) to create their own formal systems and reason about it.
  \item To integrate operational semantics with in formal system. Hence, be able represent the entire logical system with in just one framework.
  \item To show that most of formal systems have a similar meta-structure which can be implemented using common framework.
  \item To show advantages of visualised proof assistant over traditional one.
\end{itemize}

\section{Achievement}
\begin{itemize}
\item Finished designing phometa specification in such a way to keep it simple yet be able
  to produce a complex proof.
\item Finished implementing phometa. All of basic functionality is working.
\item Encoded propositional logic and typed lambda calculus in phometa.
\item Write a tutorial for newcomers to use phometa.
\end{itemize}

\end{document}