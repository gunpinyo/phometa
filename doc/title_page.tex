%%%%%%%%%%%%%%%%%%%%%%%%%%%%%%%%%%%%%%%%%
% University Assignment Title Page
% LaTeX Template
% Version 1.0 (27/12/12)
%
% This template has been downloaded from:
% http://www.LaTeXTemplates.com
%
% Original author:
% WikiBooks (http://en.wikibooks.org/wiki/LaTeX/Title_Creation)
%
% License:
% CC BY-NC-SA 3.0 (http://creativecommons.org/licenses/by-nc-sa/3.0/)
%
% Instructions for using this template:
% This title page is capable of being compiled as is. This is not useful for
% including it in another document. To do this, you have two options:
%
% 1) Copy/paste everything between \begin{document} and \end{document}
% starting at \begin{titlepage} and paste this into another LaTeX file where you
% want your title page.
% OR
% 2) Remove everything outside the \begin{titlepage} and \end{titlepage} and
% move this file to the same directory as the LaTeX file you wish to add it to.
% Then add \input{./title_page_1.tex} to your LaTeX file where you want your
% title page.
%
%%%%%%%%%%%%%%%%%%%%%%%%%%%%%%%%%%%%%%%%%

%%%%%%%%%%%%%%%%%%%%%%%%%%%%%%%%%%%%%%%%%
% I (Gun Pinyo) modify this template a little bit for individual use

%----------------------------------------------------------------------------------------
% PACKAGES AND OTHER DOCUMENT CONFIGURATIONS
%----------------------------------------------------------------------------------------

\documentclass[master.tex]{subfiles}

\begin{document}

\begin{titlepage}

\newcommand{\HRule}{\rule{\linewidth}{0.5mm}} % Defines a new command for the horizontal lines, change thickness here

\center % Center everything on the page

%----------------------------------------------------------------------------------------
% HEADING SECTIONS
%----------------------------------------------------------------------------------------

\includegraphics[scale=0.45]{Imperial_College_London}\\[1cm] % Include a department/university logo - this will require the graphicx package

\textsc{\LARGE Department of Computing}\\[2cm] % Major heading such as course name

%----------------------------------------------------------------------------------------
% TITLE SECTION
%----------------------------------------------------------------------------------------

\HRule\ \\ [0.4cm]

{ \huge \bfseries Phometa --- a visualised proof assistant}\\[0.4cm] % Title of your document
{ \huge \bfseries that build a formal system and prove }\\[0.4cm] % Title of your document
{ \huge \bfseries its theorems using derivation trees }\\[0.4cm] % Title of your document

\HRule\ \\ [1.5cm]

%----------------------------------------------------------------------------------------
% AUTHOR SECTION
%----------------------------------------------------------------------------------------

\begin{minipage}{0.3\textwidth}
\begin{flushleft} \Large
\emph{Author:}\\
Gun \textsc{Pinyo} % Your name
\end{flushleft}
\end{minipage}
~
\begin{minipage}{0.5\textwidth}
\begin{flushright} \Large
\emph{Supervisor:} \\
Dr. Krysia \textsc{Broda} \\ [0.5cm] % Supervisor's Name
\emph{Second Marker:} \\
Prof. Alessio \textsc{R. Lomuscio} \\ % Second Marker's Name
\end{flushright}
\end{minipage}\\[2cm]

% If you don't want a supervisor, uncomment the two lines below and remove the section above
%\Large \emph{Author:}\\
%John \textsc{Smith}\\[3cm] % Your name

%----------------------------------------------------------------------------------------
% DATE SECTION
%----------------------------------------------------------------------------------------

{\large \today}\\[2cm] % Date, change the \today to a set date if you want to be precise

\textsc{\large Submitted in part of fulfilment of the requirements\\ for the
  master of engineering}\\ % Name of your university/college
%----------------------------------------------------------------------------------------
% LOGO SECTION
%----------------------------------------------------------------------------------------

%----------------------------------------------------------------------------------------

\vfill % Fill the rest of the page with whitespace

\end{titlepage}

\thispagestyle{empty}

\begin{abstract}
  Manually drawing a derivation tree usually takes many iterations to be
  completed due to its layout (its width grows exponentially to its height) and
  variables being rewrite (by unification when derivation rule is applied). Even
  when the tree is completed, there are nothing to guarantee that the tree is
  error free.

  This thesis describes \emph{Phometa} which is a proof assistant that
  allows user to create a formal system and prove its theorems using derivation
  trees. Fundamentally, Phometa consists of three kinds of node which are
  \emph{Grammar} (Backus-Naur Form), \emph{Rule} (derivation rule), and
  \emph{Theorem} (derivation tree).

  It can be used as educational platform for students to learn certain formal
  systems provided in standard library. Alternatively, it also can be used as
  experimental sandbox where user implements their own formal system and try to
  reason about it.

  Phometa is a web application so components such as terms and derivation trees
  can be rendered nicely in web browser and users can interact with these
  components directly by clicking button or pressing keyboard shortcut.
  Visualisation also allows Phometa to have certain features that text-based
  proof assistants couldn't have, for example, nested underlines can be used to
  group terms instead of brackets, input method of terms can be controlled in
  such a way that ill-from terms couldn't be created, and so on.

  Phometa has been designed and been implemented in such a way that it is
  powerful enough to completely replace derivation-tree's manually-drawing, and
  easy enough to be used by anyone. Its standard library also include famous
  formal systems such as \emph{Simple Arithmetic}, \emph{Propositional Logic},
  and \emph{Typed Lambda Calculus}. This shows that Phometa is generic enough to
  handle most of formal systems out there.

\end{abstract}

\renewcommand{\abstractname}{Acknowledgements}
\begin{abstract}

  First and most importantly, I would like to thank my supervisor, Dr Krysia
  Broda, for her constant dedication to supporting me on both project skill and
  mental advise.

I also would like to thank my second marker, Prof. Alessio R. Lomuscio, for his
invaluable advice on interim report.

I also would like to thank many of Imperial's computing students, for their
feedback on my project.

I also would like to thank Evan Czaplicki and the rest of Elm community, for
their ambition and effort to develop Elm --- a wonderful front-end functional
reactive language that Phometa is built on top of.

And finally, I also would like to thank my parents for their unconditional love
and continuous support on everything since I was born.

\end{abstract}


\end{document}