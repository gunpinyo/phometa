\documentclass[master.tex]{subfiles}
\begin{document}

\chapter{Evaluation}

\section{Users Feedback --- discuss with friends}

On the $25^{th}$ of May 2016, it was the first day of project fair where
students can demonstrate their work to other students and get a feedback so I
went there and discuss about our projects. At this stage, the implementation is
finished with Simple Arithmetic and Propositional Logic included in the standard library.

I started showing my project by explaining about phometa background and Simple
Arithmetic using chapter \ref{chap:background} and
\ref{chap:example_simple_arithmetic} on this report. Then I asked them do to
exercises on chapter \ref{chap:example_simple_arithmetic} by having me as
helper. All of them understood phometa and was able to proof a theorem. Finally,
I asked them to try Propositional Logic, some of them really interest but of
them didn't want to.

From my observation, all of them were comfortable to proof by clicking options
from keymap pane rather than using keyboard shortcut. They also forgot to use
searching pattern to select options faster.

There were a few parts of user interface that were not trivial enough, they
needed to ask me what to do next, this should be fine if user have time to read
whole tutorial.

On the bright side, most of them said that they really like the way that
underlines was use to group sub-term rather than brackets (although they needed
some time to familiar with it), they also said that the proof is quite easy to
read and it will benefit newcomer.

There were several improvements that they suggest. Some of suggestions
were easy to change (e.g.\ theorems should state its goal on header as well) so
I changed it already. Some of other suggestions were quite big (e.g.\ make it
mobile friendly and have a proper server) which can be considered as future
works. We also managed to find some bugs\footnote{These bug are
  related Html and CSS rendering i.e.\ they are not related to phometa internal.} that I never found before,
this gave me an opportunity to fix it in time.

\section{Users feedback --- discuss with junior students}

\section{Professional Feedback}

\section{Strengths and Limitations}

% Evaluation will split into 2 parts as the following

% \section{Users Experience}

% As phometa is implementation based project. One way to get a truthful result is to do a survey. I will recruit several volunteers (mainly as other students) who interest logic to do the following

% \begin{enumerate}
%     \item do the tutorial that I have prepared in advance. In the tutorial, it will teach volunteer how to prove a theorem using traditional derivation tree on pen-and-paper then teach how to do the same thing with phometa. So they will know all of necessary background.
%     \item prove 6 easy-to-prove propositions of first-order logic, measure the time taken for each question. Half of questions will be proved via pen-and-paper method then proved via phometa. The other half are vice versa. So there is no bias between pen-and-paper and phometa.
%     \item create one formal system of their choice, and prove something with it.
%     \item do a questionnaire by rating 1 (Strongly disagree) to 5 (Strongly agree) on the following statement
%     \begin{itemize}
%          \item Phometa is easier use than pen-and-paper method.
%          \item Phometa take less time to prove theorem than pen-and-paper method.
%          \item Phometa is responsive-in-time enough to be interactive proof assistant.
%          \item I enjoy proving theorem on phometa.
%          \item I enjoy create a new formal system on phometa.
%          \item If I have a new idea related to formal system. I will use phometa to clarify it.
%          \item If I have know someone who interest on this area, I will recommend them to try phometa.
%          \item If I have time, I wish to contribute on phometa project.
%     \end{itemize}
% \end{enumerate}

% The entire survey for each volunteer should take 1 or 2 hours depends on their prior knowledge on this topic. I believe that it is not that hard to find a people to do this survey since this would be fun enough to keep concentration for 1 or 2 hours. Moveover, I will also mansion all of volunteers names (if they not say otherwise) as an honourable contribution on evaluation chapter.

% \section{Implementing several famous formal systems}
% Successfully implemented many famous formal systems into phometa could be used
% as indication to prove that phometa is useful.

% \section{Experts Feedback}

% I will sent phometa to a few people who expert on proof assistant and formal system to comment. This could give another aspect of evaluation from users experience. If both of parts has positive outcome, this means phometa succeed as a proof assistant.

\end{document}