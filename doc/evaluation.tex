\documentclass[master.tex]{subfiles}
\begin{document}

\chapter{Evaluation}

\section{Users Feedback --- discuss with friends}

On the $25^{th}$ of May 2016, it was the first day of project fair where
students can demonstrate their work to other students and get a feedback so I
went there and discuss about our projects. At this stage, the implementation is
finished with Simple Arithmetic and Propositional Logic included in the standard library.

I started showing my project by explaining about Phometa background and Simple
Arithmetic using chapter \ref{chap:background} and
\ref{chap:example_simple_arithmetic} on this report. Then I asked them do to
exercises on chapter \ref{chap:example_simple_arithmetic} by having me as
helper. All of them understood Phometa and was able to proof a theorem. Finally,
I asked them to try Propositional Logic, some of them really interest but of
them didn't want to.

From my observation, all of them were comfortable to proof by clicking options
from keymap pane rather than using keyboard shortcut. They also forgot to use
searching pattern to select options faster.

There were a few parts of user interface that were not trivial enough, they
needed to ask me what to do next, this should be fine if user have time to read
whole tutorial.

On the bright side, most of them said that they really like the way that
underlines was use to group sub-term rather than brackets (although they needed
some time to familiar with it), they also said that the proof is quite easy to
read and it will benefit newcomer.

There were several improvements that they suggest. Some of suggestions
were easy to change (e.g.\ theorems should state its goal on header as well) so
I changed it already. Some of other suggestions were quite big (e.g.\ make it
mobile friendly and have a proper server) which can be considered as future
works. We also managed to find some bugs\footnote{These bug are
  related Html and CSS rendering i.e.\ they are not related to Phometa internal.} that I never found before,
this gave me an opportunity to fix it in time.

\section{Users feedback --- discuss with junior students}
TODO: Wait until $1^{st}$ of May.

\section{Professional Feedback}
TODO: I am not sure should I exclude this section or not

\section{Strengths}

\begin{itemize}
\item Phometa specification itself is more powerful than traditional derivation
  system because it has extra features such as cascade premise and meta-level
  reduction. Thus, be able to support more formal systems than traditional one.
\item It has less steep learning curve than mainstream proof assistants because
  the specification is small enough for user to learn in short time and all of
  component are diagram based which is easier to understand than sequence of
  characters.
\item If a term can be construct, it is guaranteed to be well form. And if it is
  a goal of complete theorem (or lemma) it is definitely valid based on
  soundness on rules on that formal systems.
\item The repository of phometa is always in consistent state. Phometa is quite
  caution when the repository is being modified, for example, theorem can apply
  only a rule or a lemma that has been locked i.e.\ it is impossible that its
  dependencies will be changed, another example is when a node is deleted,
  phometa will delete all of node that depend on it as well\footnote{Of course,
    it will ask for confirmation first whether user want delete all of these or
    not.}. This is opposite to text-based proof assistants where user have full
  control over repository, if the repository is in inconsistent, the compiler
  will rise an error and user can fix it.
\item Lemmas allow reuse of proofs so no need for duplication. User can select
  to do forward style proving (lots of small lemmas as steps of a proof) or
  backward style proving (a few big theorems).
\item It supports unicode input method and doesn't have reserved words so formal
  system can be constructed in more mathematical friendly environment.
\item It is web-application so it can run on any machine that support web
  browser. One might argue that it required Python for back-end but most of
  machine support Python out of the box anyway.
\end{itemize}

\section{Limitation}

\begin{itemize}
\item It is hard to extend a formal system at the moment because Phometa doesn't
  allow grammar to inherit choices from another grammar. If user want to extend
  a formal system, they need to create a new one from scratch. For example,
  first order logic cannot be built from an existing propositional logic. If
  user build grammars of first order logic from scratch, existing propositional
  logic rules cannot be extended to support first order logic anyway.
\item Phometa doesn't support automation well i.e.\ when user construct a proof,
  they need to tell which rule or which lemma will be used explicitly. Guessing
  each step and automating the tree is possible, mainstream proof assistants
  such as Coq and Isabelle have done it, however, it requires lots of heuristics
  and cleaver tricks, this is unrealistic to implement due to project time
  frame but it is good consideration for future work.
\item Each web-browser supports different set of keyboard shortcuts. It is very
  hard for Phometa to find such keystrokes that are not visible characters and
  not keyboard shortcuts of any web-browser. So I end-up using \pkbd{Alt}
  combined with a visible character to create Phometa shortcut. This might have
  unwanted side effect but at least it work reasonably well with Google
  Chrome\footnote{To be precise, Chromium web-browser.} under a condition that
  the window containing Phometa have only one tab, so it will not suffer from
  \pkbd{Alt-{1..9}} are using for switch tab. However, this is not such a
  serious problem since user can always click a command in keymap pane directly.
\item The entire repository must be loaded into Phometa when it starts. This
  impacts scalability where repository is large since JavaScript can run out of
  memory. This is not usually a problem of text-based proof assistants since it
  verify a theorems one by one and doesn't need to put everything in memory at
  once.
\item Phometa required user to start a local server for individual use. It
  doesn't have a proper server where user can enter a link use it directly. To
  implement such a proper server, it requires user accounts and database to
  manage users repositories, although it is possible to implement but it seems
  to overkill method and doesn't match project objective, hence it has lower
  priority than other feature and hasn't been done.
\item Directly modify \texttt{repository.json} before it is loaded into Phometa
  could result in undefined behaviour. This is because Phometa currently doesn't
  have mechanism to verify consistency of repository before it will be loaded.
  This shouldn't cause any problem if user only interact with repository via
  Phometa interface and not try to hack repository file directly.
\end{itemize}

\end{document}