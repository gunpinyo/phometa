
%%% Local Variables:
%%% TeX-master: t
%%% coding: utf-8
%%% mode: latex
%%% End:

\documentclass[11pt, a4paper]{article}

\usepackage{config}

\begin{document}

\section{title}

Good morning everybody, today, it is an honour for me to present my dedicated
proof assistant named ``Phometa'' which is a tool that can be used to construct
a formal system and prove its theorems.

\section{Derivation systems can reason many formal systems.}

Derivation system is capable to reason about many formal systems. For example,
propositional logic, we can prove that [point] $p \wedge q$ holds under
assumption that [point] $p$ and $p \rightarrow q$ hold, using grammars and
derivation rules of simple arithmetic.

[next page]

The similar things also happen in simple arithmetic and typed lambda calculus as
well.

\section{Problem of drawing a derivation tree.}

However, drawing a derivation tree manually in a paper is painful and
tedious. This is because its width grows exponentially to its height so it is
hard to fit in a paper, if a mistake or unification occur, the entire tree might
need to be redrawn again, and most importantly, it is doesn't have a precise way
to verify that a tree is error free.

So, I decided to build ``Phometa'' to encode any formal system that support
derivation tree in order to solve the problem above.

\section{Screenshot of  ``Phometa'' in action}

[point to screen] Here is a screenshot that shows Phometa in action, you can see
that the derivation tree from the first slide can be encoded into Phometa
systematically and the tree doesn't have problem of its width grows
exponentially to its height anymore.

We also exploit the power of visualisation such as using underlines to group
sub-terms instead of brackets.

\section{Formal system ingredient - Grammar (Backus-Nour Form)}

To encode any formal system, we need to defines its grammars and its derivation
rules. For propositional logic, we can define grammars starting with ``prop''
[point to bnf of prop] which is, in fact Backus-Nour Form. Each alternative can
be mapped into Phometa grammar node [point to grammar node of ``prop''] of
property ``choice''. Then a term [point to the top of parse tree] can be
instantiated to one of these alternatives. In addition, Phometa also allow a
term to be Phometa also allow Grammar node also has a property
``metavar\_regex'' that allow ``prop'' to have meta variables under a condition
that it must follows this [point to regex] regex pattern.

Now let see node grammar of ``atom'', we use literals instead meta variables
here and the different between them is that a meta variable can match any term
of the same grammar but a literal only match to itself.

\section{Grammar (Backus-Nour Form)}

We also need a grammar for ``context'' which is a set of assumptions and
``judgement'' which is a pair of ``context'' and ``prop'' now we can show the
parse tree of the judgement in the screenshot on the previous slide.

\section{Formal system ingredient - Rule (Derivation Rule)}

Once we define grammars, we can define derivation rules. Nevertheless
propositional logic has so many rules so I will just explain one of them which
is ``imply-elim''. This rule state [point to derivation tree] that we can prove
that $B$ holds if $A \rightarrow B$ and $A$ hold using the same assumptions.
Premises and conclusion of this derivation tree can be mapped directly into this
rule node using properties ``premise'' and ``conclusion'' [point to those
keywords respectively].

A rule will be used mainly to build a tree, for example [point to the
sub-theorem], we have an unfinished tree with a goal state that $q$ holds under
assumptions that $p$ and $p \rightarrow q$ hold, we can apply rule
``imply-elim'', the rule will \emph{pattern match} its conclusion against the
goal, this results in $\gamma$ as list of $p$ and $p \rightarrow q$, and $B$ as
$q$. This pattern matching result will substitute each of premises to get
further sub-goals that can be fulfilled similar method to this goal.

Some meta variable such as $A$ appears in premises but not in the conclusion, so
it cannot be substituted by pattern matching result, this kind of meta variable
is called ``parameter'' where we need to specify explicitly every time when we
apply a rule. This is become clearer when we go though during demonstration.

\section{Formal system ingredient - Theorem (Derivation Tree)}

The last ingredient theorem node which are, in fact, derivation trees.

Each tree consists of goal (on r.h.s.), a rule (on l.h.s.), and possibly further
sub-trees for sub-goals (above).

\section{Demonstration}

Now, I will show you a video that demonstrate Phometa, this should take around
10 minutes before we resume back to the slides.

\section{Cascade Premise --- Hypothesis Rule}

According to the demonstration, hypothesis rule is clearly not a ordinary
inference rule because it can penetrate throughout of assumptions. In fact
hypothesis rule use spacial premise called ``Cascade Premise'', which will try
to call its first sub-rule, if success then include sub-goals of sub-rules and
its sub-goals, otherwise cascade down to lower sub-rule and repeat again. So for
hypothesis it call hypothesis-base that can prove the judgement that has the
conclusion as the the last assumption, if it fails, remove its last assumption
and call itself again, until the right assumption is found.

\section{Meta-Reduction --- Context Manipulation}

Another feature that is missing out from the demonstration is meta-reduction,
well, context is supposed to be a set but we encode it as a list, hence we
should be able to freely permute and duplicate assumptions in context around
implicitly. To archive this ability, we can define a rule that has exactly one
premise and set the flag ``allow\_reduction'', so we can manipulate term in active
sub-goals of a theorem by clicking one of its sub-term, then we can select the
corresponded rule in the keymap plan, then the conclusion will pattern match
against the target term and replace it with substituted version of premise. For
example this judgement as this context as sub-term so if we use
``context-commutative'' the context will swap the last two assumption.

\section{Implementation --- Elm language and its Signals}

The main language that is used to implement Phometa is Elm which is haskell-like
programming language that compile to JavaScript. One of the most attractive
feature of Elm is Signal which is an object that change its value over the time.
For example \texttt{Mouse.isDown} is a Signal of type \texttt{Bool} and its
value will be changed whenever the mouse is clicked or released.

We can manipulate signals around [point] as described in dependency graph,
eventually, we will get a signal of type HTML which can be displayed in the web
browser, for example, the code on the screen will be complied to Html that say
``foo'', if we click mouse on the screen, it will change to ``bar''. It
will go to ``To click again'' again once we release the mouse.

\section{Implementation --- Model-Controller-View (MCV)}

However, programming in practical require internal state of the program, so we
can use foldp or ``fold from the past'' to accumulated an incoming event
with the previous state of the program. This results in MCV pattern
and the entire implementation is split into 3 questions.

first, how to model the global state of the program

second, how to write a function that merge an incoming event with the past model

and third, how to and write a function to convert the global state of the
program into html.

Therefore, the logical and user interface of the programming is completely
separated, this combine with the fact that it is functional programming can
greatly reduce the possibility of bugs and errors.

Please note that, this is a simplified version of phometa main entry, the
real thing need to deal with outside world as well such as communicating with
the server to load and save the proof repository.

\section{Strengths, Limitation, and Future Work}
Phometa has additional features over the derivation system such as prove by
lemma, cascade premise, and meta-reduction.

It is also easier to learn compared to popular proof assistants such as Coq or Agda.

And ultimately, it is impossible to construct an invalid proof.

But on the down side, phometa need to load everything when it starts, this will
become problematic when the repository is large.

There are also several other limitation that can be considered as future works
for example, some proving step should be more automatic, nodes should be
exportable to latex source code, and importation among modules.

\section{Achievement \& Conclusion}

As the result of this project, I have successfully designed and implemented a
proof assistant to solve the problem regarding to derivation tree at the
beginning of this presentations..

I also showed that this program work well for most of formal systems by writing
several famous formal systems in standard library.

I also showed that the program is easy to use by anybody by
writing a tutorial that is short and doesn't require prior knowledge.

In conclusion, I believe that Phometa is powerful and easy enough to be used as
a replacement of traditional derivation tree. This is at the very least, can
help student to do their coursework that need to draw derivation trees.

% \section{Demo}
% \begin{itemize}
% \item Overview of Phometa's user interface.\\

%   Good morning everybody, in this video, I will walk through some basic features
%   and usability of Phometa. Well, Let's start.

%   Once the program is opened in a web-browser, you will see the welcome page
%   [circle welcome] alongside with the keymap pane that show every possible
%   keyboard shortcut or you can click it directly as well. Phometa also have the
%   navigation bar [circle package pane] that represents global structure of proof
%   repository which consists of packages and modules. [fold arith] This initial
%   proof repository consists 3 formal systems which are simple arithmetic,
%   propositional logic, and simply typed lambda calculus. [unfold propositional
%   logic] If we click on propositional logic you will see a list of node [scroll
%   down] that can be either comment nodes [click modules description], grammar
%   nodes [click prop], rule nodes [click hypo and imply-elim on package pane],
%   and theorem nodes [example-theorem-1 on package pane].

%   Now we want to prove that $p \wedge q$ holds under assumptions of $p$ and $p
%   \rightarrow q$, we can do this by clicking ``add theorem'' and enter the name
%   of this theorem, then we can construct a term that we want to prove, in this
%   case it is ``Judgement'' [circle and click], then it will split into
%   ``context'' and ``prop'', for context we need 2 assumptions so expand it twice
%   before terminate it with epsilon, then the cursor will focus at the first
%   assumption, we can select it as ``atom'' then type ``p'' and hit enter, for
%   the second assumption we click ``imply'' ``atom'' ``p'' ``atom'' type ``p'',
%   the conclusion is ``and'' ``atom'' ``p'' ``atom'' ``q''.

%   once the goal of this theorem is fully constructed, we can apply a rule to
%   this goal. In order to make it clearer will split screen first [alt-x 8], now
%   we can see on going proof [click it] and the rule, in this case ``and-intro''
%   [click it], we can apply a rule by click ``prove by rule'' then search a rule
%   by input the filter until be find the rule and hit enter. This will fulfil the
%   main goal and generate 2 further sub-goals according to pattern matching on
%   derivation rule. Now the conclusion of the first sub-goal is one of its
%   assumption so we just apply ``hypothesis'', for the second sub-goal we will
%   apply ``imply-elim'' [select ``imply elim''] now it will ask that what ``A''
%   should be, if we see its definition [click ``imply-elim'' on another pane]
%   ``A'' is the antecedent of implication so we want it to be ``atom'' ``p''  in
%   this case, and now we get further sub-goals, both of them is in assumptions so
%   apply by ``hypothesis'' and the theorem is completed.

%   But wait how can ``hypothesis'' penetrate assumptions like that. Well we can
%   see it definition, [open ``hypothesis-base'' and ``hypothesis'']. First we
%   have ``hypothesis-base'', this rule proved any judgement that has the
%   conclusion as the last assumption and generate no further sub-goal,
%   ``hypothesis'' make use of this calling ``hypothesis-base'' if this success
%   the whole application success, otherwise it cut the last assumption and call
%   itself recursively until ``hypothesis-base'' success.

% \item Prove that \term{demo-term-2-1} is a valid \term{demo-pgmr-judgement}.\\

% TODO: use law of exclude middle
% TODO: show meta reduction

% TODO: talk about focus / unfocus

% \item Prove that \term{demo-term-4-1} is a valid \term{demo-pgmr-prop}.\\

% \item Extend Propositional Logic to support equivalence.\\

% \item Skim though Simple Arithmetic and Typed Lambda Calculus.
% \end{itemize}
\end{document}
