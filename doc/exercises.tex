\documentclass[master.tex]{subfiles}
\begin{document}

\chapter{Exercises on Examples Formal Systems}
\label{chap:excercises}

\section{Simple Arithmetic}

\begin{itemize}
\item Create a theorem and proof each of the following
  \begin{itemize}
  \item \term{arith-ex-1-1}
  \item \term{arith-ex-1-2}
  \item \term{arith-ex-1-3}
  \end{itemize}
\item Create a theorem of your own choice and proof it.
\item (Challenge) Extend Simple Arithmetic to support the following
  \begin{itemize}
  \item addition and multiplication identity.
  \item addition and multiplication idempotent.
  \item inequality.
  \end{itemize}
  You may need to create a new grammar or rules for this, please see section
  \ref{sec:how_to_built_grammars_and_rules} in the next chapter.
\end{itemize}

\section{Propositional Logic}

Credit: Some of material here modified from tutorial 3, 4, and 5 of ``140 -
Logic'' (first year course), Department of Computing, Imperial College London.
Thank you Prof Ian Hodkinson and Dr Krysia Broda for this.

\begin{itemize}
\item Create a theorem and proof each of the following
  \begin{itemize}
  \item \term{prop-ex-1-a}
  \item \term{prop-ex-1-b} (reminder: this is \pgmr{Prop} not \pgmr{Judgement})
  \item \term{prop-ex-1-c}
  \item \term{prop-ex-1-d}
  \item \term{prop-ex-1-e}
  \item \term{prop-ex-1-f}
  \item \term{prop-ex-1-g}
  \item \term{prop-ex-2-a}
  \item \term{prop-ex-2-b}
  \item \term{prop-ex-2-c}
  \item \term{prop-ex-2-d}
  \item \term{prop-ex-2-e}
  \item \term{prop-ex-2-f}
  \item \term{prop-ex-2-g}
  \item \term{prop-ex-2-h}
  \item \term{prop-ex-2-i}
  \item \term{prop-ex-2-j}
  \end{itemize}

\newpage

\item Equivalence of Propositional Logic could be written in the form ($A \equiv
  B$) stated that, $A$ holds if and only if $B$ holds. Please
  introduce new a grammar \pgmr{Equivalence} and write \prule{equiv-intro}
  similar section \ref{sec:validity_of_proposition} and prove the following
  \begin{itemize}
  \item \term{prop-ex-eqv-1}
  \item \term{prop-ex-eqv-2}
  \item \term{prop-ex-eqv-3}
  \item \term{prop-ex-eqv-4}
  \item \term{prop-ex-eqv-5}
  \item \term{prop-ex-eqv-6}
  \item \term{prop-ex-eqv-7}
  \item \term{prop-ex-eqv-8}
  \item \term{prop-ex-eqv-9}
  \item \term{prop-ex-eqv-10}
  \item \term{prop-ex-eqv-11}
  \item \term{prop-ex-eqv-12}
  \item \term{prop-ex-eqv-13}
  \item \term{prop-ex-eqv-14}
  \item \term{prop-ex-eqv-15}
  \end{itemize}
\item (Challenge) Extend this Propositional Logic to become First Order Logic.
\end{itemize}

\end{document}
