\documentclass[master.tex]{subfiles}
\begin{document}

\chapter{Related Work}

There are many proof assistants available out there, each of them rely on
slightly different meta-theory. We can separate proof assistants into 2
categories which are text-base proof assistants and visualised proof assistants.

\section{Text-Base Proof Assistants}

Text-base proof assistants are similar to programming language where user writes
everything in text-files and compile it, if the compilation is successful, then
the proofs are correct. User can freely manipulate these text-files, hence,
easier to write a complex proof. In addition, most of proof assistants have a
plug-in to mainstream text editor, so user can use their favourite text editor
with full performance.

There are several mainstream text-base proof assistants that worth mentioning

\subsection{Coq}
Coq\supercite{coq-official-website} is one the most famous proof assistants. It
is based on the Calculus of Inductive Constructions (CIC)\footcite{CIC is itself
  is developed alongside Coq.} developed by Thierry
Coquand\supercite{thierry-coquand-homepage}.

Coq has customisable tactics which are commands that transform goal into
smaller-sub goal (if any), this makes proving process become faster compared to
other proof assistants. In contrast, tactics reduce readability, reader might
need to replay each tactic step by step in order to understand a proof
completely.

Coq is very mature, it has been developed since 1984. Hence, it is reliable and
has lots of libraries supported.

In term of editor, most people use Proof
General\supercite{proof-general-official-website} which is a plugin on
Emacs\footnote{Proof General also other proof assistants such as Isabelle and PhoX}.
Nevertheless, Coq has its own editor called
CoqIde\supercite{coqide-official-website} that newcomers can use without
learning Emacs.

\begin{figure}[H]
    \centering
    \includegraphics[width=\textwidth]{related-work-coq}
    \caption{Screenshot of Coq (using CoqIde, Credit:
      \cite{coq-official-website}) --- The left pane is file content and Upper
      right pane is the current goal which is changed depending on where the
      cursor point on file content.}
\label{fig:related-work-coq}
\end{figure}

\subsection{Agda}
Adga\supercite{agda-official-website} is (dependently typed) functional
programming which can be seen as a proof assistant as well. It is based on
Unified Theory of Dependent
Types\supercite{norell:thesis}\supercite{Luo:1994:CRT:184757} similar to Martin
Lof Type Theory.

Its proving technique is relies on Curry-Howard correspondence
which state that there is duality between computer programs and mathematical
proofs\supercite{curry-howard-correspondence}, for example function corresponded
to implication, product type corresponded to logical implication.

Agda is suitable for reasoning about functional programs because we can write a
program and prove that certain properties of a function hold using the same
language. This is feasible since a proof is just a function due to Curry-Howard
correspondence.

Agda has less steep learning curve compared other proof assistants such as Coq.
This is because user doesn't need to learn about proving
system since it is the same as programming. In contrast, it doesn't have
fancy tactic system so proving process is slower.

In term of popularity, it is less popular than Coq, however, some project such
as Homotopy Type Theory\supercite{hott-coq-repo}\supercite{hott-agda-repo} use
Agda as alternative experiments to Coq.

In term of editor, Agda as its own plugin for Emacs which is very nice but user
need to be familiar to Emacs before using it. There is no alternative plugin to
other editor.

\begin{figure}[H]
    \centering
    \includegraphics[width=0.6\textwidth]{related-work-agda}
    \caption{Screenshot of Agda --- Credit: an example in Agda standard library,
    removing comment out to save space.}
\label{fig:related-work-agda}
\end{figure}


\subsection{Isabelle}

Isabelle\supercite{isabelle-official-website} is generic proof assistant which
allows user to express mathematical formulae in a formal language and provide a
tool to prove something about it. There are many systems that Isabelle supports
but the most widespread one is \emph{Isabelle/HOL} which provide a higher order
logic environment that is ready for a big application.

One of the main advantage of Isabelle is its readability, the proofs will be
constructed by a language called \emph{Isar} which is designed in such a way
that it could be read easily by both of computers and humans. Another advantage
of Isabelle is that some part of a proof can be automatically proven, this
improve user productivity.

In term of editor, Isabelle has default user interface and Prover IDE called
\emph{Isabelle/jEdit} which is based on jEdit and Isabelle/Scala.

\begin{figure}[H]
    \centering
    \includegraphics[width=0.8\textwidth]{related-work-isabelle}
    \caption{Screenshot of Isabelle (using Isabelle/jEdit, Credit:
      \cite{isabelle-official-website}) --- The top-left pane shows file content
      and the right pane shows content structure.}
\label{fig:related-work-isabelle}
\end{figure}

% \subsection{Lean}

% Lean\supercite{lean-offical-website} is a relatively new interactive theorem
% prover\footnote{The Lean project was launched by Leonardo de Moura at Microsoft
%   Research in 2013}, because of this it can borrow some

\section{Visualised Proof Assistant}
Visualised proof assistants show proofs and related contents in graphical way,
it also allow user interact with proofs mainly by clicking which make them
easier to use by newcomers, nevertheless, user cannot fully manipulate the proof
as they have limited choices of input method, this makes it is harder have
some advance features.

There are so visualised proof assistants so there, I will show a few of them to
illustrate what visualised proof assistants look like.

\subsection{Logitext}
Logitext\supercite{logitext-official-website} is a web-based proof assistant for
\emph{first-order classical logic} using the \emph{sequent calculus}. The main
intention is to teach students about \emph{Gentzen trees} which one of a way to
struct derivation system. Logitext uses Coq internally to check validity of
proof steps.

Logitext is very easy use, user usually only need to click a logical connector
on the goal itself to construct a derivation tree. However, it only supports
sequent calculus for first-order classical logic or propositional intuitionistic
logic and user cannot extend the system.

\begin{figure}[H]
    \centering
    \includegraphics[width=0.5\textwidth]{related-work-logitext-1}
    \includegraphics[width=0.6\textwidth]{related-work-logitext-2}
    \caption{Screenshots of Logitext --- You can see that a rule could be invoke
      by just clicking a logical connector.}
\label{fig:related-work-logitext}
\end{figure}

\subsection{Panda}
Panda\supercite{panda-official-website}\supercite{Gasquet2011} (\textbf{P}roof
\textbf{A}ssistant for \textbf{N}atural \textbf{D}eduction for \textbf{A}ll) is
graphical proof assistant that can be used to prove first order logic using
Gentzen style's natural deduction.

User will interact with it mainly by drag and drop. The main advantage is its
tutorial which is well integrate with the actual program itself.

\begin{figure}[H]
    \centering
    \includegraphics[width=0.8\textwidth]{related-work-panda}
    \caption{Screenshots of Panda --- User can expand the current tree by
      selecting a rule on the left pane and can drag and drop formula around to
      connect it to another one.}
\label{fig:related-work-panda}
\end{figure}

\newpage

\subsection{Pandora}
Pandora\supercite{pandora-official-website}\supercite{Broda2007PandoraAR}
(\textbf{P}roof \textbf{A}ssistant for \textbf{N}atural \textbf{D}eduction using
\textbf{O}rganised \textbf{R}ectangular \textbf{A}reas) is graphical proof
assistant that can be used to prove first order logic using Fetch style's
natural deduction.

The usage is quite similar to Panda, but it use boxes (Fetch style) rather than
derivation tree (Gentzen style). It also have a comprehensive document for
newcomer as well.

\begin{figure}[H]
    \centering
    \includegraphics[width=0.8\textwidth]{related-work-pandora}
    \caption{Screenshots of Pandora (Credit: \cite{Broda2007PandoraAR}) ---
      The left pane shows current proof, user can use a rule by clicking on the
      upper pane.}
\label{fig:related-work-pandora}
\end{figure}

\end{document}